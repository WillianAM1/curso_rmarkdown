% Options for packages loaded elsewhere
\PassOptionsToPackage{unicode}{hyperref}
\PassOptionsToPackage{hyphens}{url}
%
\documentclass[
]{book}
\usepackage{lmodern}
\usepackage{amssymb,amsmath}
\usepackage{ifxetex,ifluatex}
\ifnum 0\ifxetex 1\fi\ifluatex 1\fi=0 % if pdftex
  \usepackage[T1]{fontenc}
  \usepackage[utf8]{inputenc}
  \usepackage{textcomp} % provide euro and other symbols
\else % if luatex or xetex
  \usepackage{unicode-math}
  \defaultfontfeatures{Scale=MatchLowercase}
  \defaultfontfeatures[\rmfamily]{Ligatures=TeX,Scale=1}
\fi
% Use upquote if available, for straight quotes in verbatim environments
\IfFileExists{upquote.sty}{\usepackage{upquote}}{}
\IfFileExists{microtype.sty}{% use microtype if available
  \usepackage[]{microtype}
  \UseMicrotypeSet[protrusion]{basicmath} % disable protrusion for tt fonts
}{}
\makeatletter
\@ifundefined{KOMAClassName}{% if non-KOMA class
  \IfFileExists{parskip.sty}{%
    \usepackage{parskip}
  }{% else
    \setlength{\parindent}{0pt}
    \setlength{\parskip}{6pt plus 2pt minus 1pt}}
}{% if KOMA class
  \KOMAoptions{parskip=half}}
\makeatother
\usepackage{xcolor}
\IfFileExists{xurl.sty}{\usepackage{xurl}}{} % add URL line breaks if available
\IfFileExists{bookmark.sty}{\usepackage{bookmark}}{\usepackage{hyperref}}
\hypersetup{
  pdftitle={Introdução ao R Markdown},
  pdfauthor={Eduardo José de Campos Lemos Júnior, Samuel Vianna Quintanilha},
  hidelinks,
  pdfcreator={LaTeX via pandoc}}
\urlstyle{same} % disable monospaced font for URLs
\usepackage{color}
\usepackage{fancyvrb}
\newcommand{\VerbBar}{|}
\newcommand{\VERB}{\Verb[commandchars=\\\{\}]}
\DefineVerbatimEnvironment{Highlighting}{Verbatim}{commandchars=\\\{\}}
% Add ',fontsize=\small' for more characters per line
\usepackage{framed}
\definecolor{shadecolor}{RGB}{248,248,248}
\newenvironment{Shaded}{\begin{snugshade}}{\end{snugshade}}
\newcommand{\AlertTok}[1]{\textcolor[rgb]{0.94,0.16,0.16}{#1}}
\newcommand{\AnnotationTok}[1]{\textcolor[rgb]{0.56,0.35,0.01}{\textbf{\textit{#1}}}}
\newcommand{\AttributeTok}[1]{\textcolor[rgb]{0.77,0.63,0.00}{#1}}
\newcommand{\BaseNTok}[1]{\textcolor[rgb]{0.00,0.00,0.81}{#1}}
\newcommand{\BuiltInTok}[1]{#1}
\newcommand{\CharTok}[1]{\textcolor[rgb]{0.31,0.60,0.02}{#1}}
\newcommand{\CommentTok}[1]{\textcolor[rgb]{0.56,0.35,0.01}{\textit{#1}}}
\newcommand{\CommentVarTok}[1]{\textcolor[rgb]{0.56,0.35,0.01}{\textbf{\textit{#1}}}}
\newcommand{\ConstantTok}[1]{\textcolor[rgb]{0.00,0.00,0.00}{#1}}
\newcommand{\ControlFlowTok}[1]{\textcolor[rgb]{0.13,0.29,0.53}{\textbf{#1}}}
\newcommand{\DataTypeTok}[1]{\textcolor[rgb]{0.13,0.29,0.53}{#1}}
\newcommand{\DecValTok}[1]{\textcolor[rgb]{0.00,0.00,0.81}{#1}}
\newcommand{\DocumentationTok}[1]{\textcolor[rgb]{0.56,0.35,0.01}{\textbf{\textit{#1}}}}
\newcommand{\ErrorTok}[1]{\textcolor[rgb]{0.64,0.00,0.00}{\textbf{#1}}}
\newcommand{\ExtensionTok}[1]{#1}
\newcommand{\FloatTok}[1]{\textcolor[rgb]{0.00,0.00,0.81}{#1}}
\newcommand{\FunctionTok}[1]{\textcolor[rgb]{0.00,0.00,0.00}{#1}}
\newcommand{\ImportTok}[1]{#1}
\newcommand{\InformationTok}[1]{\textcolor[rgb]{0.56,0.35,0.01}{\textbf{\textit{#1}}}}
\newcommand{\KeywordTok}[1]{\textcolor[rgb]{0.13,0.29,0.53}{\textbf{#1}}}
\newcommand{\NormalTok}[1]{#1}
\newcommand{\OperatorTok}[1]{\textcolor[rgb]{0.81,0.36,0.00}{\textbf{#1}}}
\newcommand{\OtherTok}[1]{\textcolor[rgb]{0.56,0.35,0.01}{#1}}
\newcommand{\PreprocessorTok}[1]{\textcolor[rgb]{0.56,0.35,0.01}{\textit{#1}}}
\newcommand{\RegionMarkerTok}[1]{#1}
\newcommand{\SpecialCharTok}[1]{\textcolor[rgb]{0.00,0.00,0.00}{#1}}
\newcommand{\SpecialStringTok}[1]{\textcolor[rgb]{0.31,0.60,0.02}{#1}}
\newcommand{\StringTok}[1]{\textcolor[rgb]{0.31,0.60,0.02}{#1}}
\newcommand{\VariableTok}[1]{\textcolor[rgb]{0.00,0.00,0.00}{#1}}
\newcommand{\VerbatimStringTok}[1]{\textcolor[rgb]{0.31,0.60,0.02}{#1}}
\newcommand{\WarningTok}[1]{\textcolor[rgb]{0.56,0.35,0.01}{\textbf{\textit{#1}}}}
\usepackage{longtable,booktabs}
% Correct order of tables after \paragraph or \subparagraph
\usepackage{etoolbox}
\makeatletter
\patchcmd\longtable{\par}{\if@noskipsec\mbox{}\fi\par}{}{}
\makeatother
% Allow footnotes in longtable head/foot
\IfFileExists{footnotehyper.sty}{\usepackage{footnotehyper}}{\usepackage{footnote}}
\makesavenoteenv{longtable}
\usepackage{graphicx}
\makeatletter
\def\maxwidth{\ifdim\Gin@nat@width>\linewidth\linewidth\else\Gin@nat@width\fi}
\def\maxheight{\ifdim\Gin@nat@height>\textheight\textheight\else\Gin@nat@height\fi}
\makeatother
% Scale images if necessary, so that they will not overflow the page
% margins by default, and it is still possible to overwrite the defaults
% using explicit options in \includegraphics[width, height, ...]{}
\setkeys{Gin}{width=\maxwidth,height=\maxheight,keepaspectratio}
% Set default figure placement to htbp
\makeatletter
\def\fps@figure{htbp}
\makeatother
\usepackage[normalem]{ulem}
% Avoid problems with \sout in headers with hyperref
\pdfstringdefDisableCommands{\renewcommand{\sout}{}}
\setlength{\emergencystretch}{3em} % prevent overfull lines
\providecommand{\tightlist}{%
  \setlength{\itemsep}{0pt}\setlength{\parskip}{0pt}}
\setcounter{secnumdepth}{5}

\title{Introdução ao R Markdown}
\author{Eduardo José de Campos Lemos Júnior, Samuel Vianna Quintanilha}
\date{2020-07-29}

\begin{document}
\maketitle

{
\setcounter{tocdepth}{1}
\tableofcontents
}
\hypertarget{intro}{%
\chapter{Introdção}\label{intro}}

\hypertarget{o-que-uxe9}{%
\section{O que é}\label{o-que-uxe9}}

Exmplicar Rmarkdown markdown e explicar a diferença

\hypertarget{possuxedveis-tipos-de-outputs}{%
\section{Possíveis tipos de Outputs}\label{possuxedveis-tipos-de-outputs}}

\begin{itemize}
\tightlist
\item
  \texttt{beamer\_presentation}
\item
  \texttt{context\_document}
\item
  \texttt{github\_document}
\item
  \texttt{html\_document}
\item
  \texttt{ioslides\_presentation}
\item
  \texttt{latex\_document}
\item
  \texttt{md\_document}
\item
  \texttt{odt\_document}
\item
  \texttt{pdf\_document}
\item
  \texttt{powerpoint\_presentation}
\item
  \texttt{rtf\_document}
\item
  \texttt{slidy\_presentation}
\item
  \texttt{word\_document}
\end{itemize}

\hypertarget{download-rmd-e-latex}{%
\section{Download (rmd e latex)}\label{download-rmd-e-latex}}

\hypertarget{windows}{%
\subsection{Windows}\label{windows}}

\hypertarget{linux}{%
\subsection{Linux}\label{linux}}

\hypertarget{macos}{%
\subsection{MacOS}\label{macos}}

\hypertarget{criando-o-primeiro-documento}{%
\section{Criando o primeiro documento}\label{criando-o-primeiro-documento}}

Mostrar como se cria um arquivo Rmd, tanto no Rstudio quanto fora

\begin{verbatim}

```r
print(2)
```

```
## [1] 2
```

\end{verbatim}

\hypertarget{sintaxe}{%
\chapter{Sintaxe}\label{sintaxe}}

\hypertarget{pruxeaambulo}{%
\section{Prêambulo}\label{pruxeaambulo}}

\begin{verbatim}
---
output: pdf_document()
---
\end{verbatim}

\hypertarget{informauxe7uxf5es-gerais-autor-titulo-data}{%
\subsection{Informações Gerais (autor titulo data\ldots)}\label{informauxe7uxf5es-gerais-autor-titulo-data}}

\hypertarget{definindo-tipo-de-output}{%
\subsection{Definindo tipo de output}\label{definindo-tipo-de-output}}

\hypertarget{sumario}{%
\subsection{Sumario}\label{sumario}}

\hypertarget{textos}{%
\section{Textos}\label{textos}}

\hypertarget{cabeuxe7alhos}{%
\subsection{cabeçalhos}\label{cabeuxe7alhos}}

\begin{verbatim}
#

##

###

####
\end{verbatim}

\hypertarget{formatauxe7uxe3o-de-textos-negrito-ituxe1lico-e-cuxf3digo}{%
\subsection{Formatação de textos (negrito, itálico e código)}\label{formatauxe7uxe3o-de-textos-negrito-ituxe1lico-e-cuxf3digo}}

\begin{verbatim}
*Negrito* _Negrito_
\end{verbatim}

\textbf{Negrito} \textbf{Negrito}

\begin{verbatim}
**Italico** __italico__
\end{verbatim}

\emph{Italico} \emph{Italico}

\begin{Shaded}
\begin{Highlighting}[]
\InformationTok{\textasciigrave{}\textasciigrave{}\textasciigrave{}\{r\}}
\InformationTok{Sys.info()}
\InformationTok{\textasciigrave{}\textasciigrave{}\textasciigrave{}}
\end{Highlighting}
\end{Shaded}

\begin{Shaded}
\begin{Highlighting}[]
\KeywordTok{Sys.Date}\NormalTok{()}
\end{Highlighting}
\end{Shaded}

\begin{verbatim}
## [1] "2020-07-29"
\end{verbatim}

\hypertarget{links}{%
\subsection{Links}\label{links}}

Inserindo links \href{https://estatsej.github.io/curso_rmarkdown}{Curso de R Markdown}

\href{https://rstudio.com/}{Rstudio}

\hypertarget{listas}{%
\subsection{Listas}\label{listas}}

\begin{verbatim}
- Exemplo
- De
- Listas
  - 1
  - 2
  - 3
  
1. primeiro
2. segundo
3. terceiro
\end{verbatim}

\begin{itemize}
\tightlist
\item
  Exemplo
\item
  De
\item
  Listas

  \begin{itemize}
  \tightlist
  \item
    1
  \item
    2
  \item
    3
  \end{itemize}
\end{itemize}

\begin{enumerate}
\def\labelenumi{\arabic{enumi}.}
\tightlist
\item
  primeiro
\item
  segundo
\item
  terceiro
\end{enumerate}

\hypertarget{inserindo-imagens}{%
\section{Inserindo Imagens}\label{inserindo-imagens}}

\begin{verbatim}
![](http://tny.im/lTZ)
\end{verbatim}

\includegraphics{http://tny.im/lTZ}

\begin{Shaded}
\begin{Highlighting}[]
\InformationTok{\textasciigrave{}\textasciigrave{}\textasciigrave{}\{r\}}
\InformationTok{knitr::include\_graphics("http://tny.im/lTZ")}
\InformationTok{\textasciigrave{}\textasciigrave{}\textasciigrave{}}
\end{Highlighting}
\end{Shaded}

\begin{Shaded}
\begin{Highlighting}[]
\NormalTok{knitr}\OperatorTok{::}\KeywordTok{include\_graphics}\NormalTok{(}\StringTok{"http://tny.im/lTZ"}\NormalTok{)}
\end{Highlighting}
\end{Shaded}

\includegraphics{http://tny.im/lTZ}

\hypertarget{exemplos}{%
\section{Exemplos}\label{exemplos}}

\href{exemplos/102-sumario_e_cabecalho.html}{Exemplo Sumário e Cabeçalhos}

\href{exemplos/103-formatacao_de_texto.html}{Exemplo formatação de textos}

\hypertarget{executando-cuxf3digo}{%
\chapter{Executando Código}\label{executando-cuxf3digo}}

\hypertarget{introduuxe7uxe3o}{%
\section{Introdução}\label{introduuxe7uxe3o}}

\begin{verbatim}
```{r}

```
\end{verbatim}

\hypertarget{flags}{%
\section{Flags}\label{flags}}

There are a large number of chunk options\index{chunk options} in \textbf{knitr} documented at \url{https://yihui.name/knitr/options}. We list a subset of them below:

\begin{itemize}
\item
  \texttt{eval}: Whether to evaluate a code chunk.
\item
  \texttt{echo}: Whether to echo the source code in the output document (someone may not prefer reading your smart source code but only results).
\item
  \texttt{results}: When set to \texttt{\textquotesingle{}hide\textquotesingle{}}, text output will be hidden; when set to \texttt{\textquotesingle{}asis\textquotesingle{}}, text output is written ``as-is'', e.g., you can write out raw Markdown text from R code (like \texttt{cat(\textquotesingle{}**Markdown**\ is\ cool.\textbackslash{}n\textquotesingle{})}). By default, text output will be wrapped in verbatim elements (typically plain code blocks).
\item
  \texttt{collapse}: Whether to merge text output and source code into a single code block in the output. This is mostly cosmetic: \texttt{collapse\ =\ TRUE} makes the output more compact, since the R source code and its text output are displayed in a single output block. The default \texttt{collapse\ =\ FALSE} means R expressions and their text output are separated into different blocks.
\item
  \texttt{warning}, \texttt{message}, and \texttt{error}: Whether to show warnings, messages, and errors in the output document. Note that if you set \texttt{error\ =\ FALSE}, \texttt{rmarkdown::render()} will halt on error in a code chunk, and the error will be displayed in the R console. Similarly, when \texttt{warning\ =\ FALSE} or \texttt{message\ =\ FALSE}, these messages will be shown in the R console.
\item
  \texttt{include}: Whether to include anything from a code chunk in the output document. When \texttt{include\ =\ FALSE}, this whole code chunk is excluded in the output, but note that it will still be evaluated if \texttt{eval\ =\ TRUE}. When you are trying to set \texttt{echo\ =\ FALSE}, \texttt{results\ =\ \textquotesingle{}hide\textquotesingle{}}, \texttt{warning\ =\ FALSE}, and \texttt{message\ =\ FALSE}, chances are you simply mean a single option \texttt{include\ =\ FALSE} instead of suppressing different types of text output individually.
\item
  \texttt{cache}: Whether to enable caching. If caching is enabled, the same code chunk will not be evaluated the next time the document is compiled (if the code chunk was not modified), which can save you time. However, I want to honestly remind you of the two hard problems in computer science (via Phil Karlton): naming things, and cache invalidation. Caching can be handy but also tricky sometimes.
\item
  \texttt{fig.width} and \texttt{fig.height}: The (graphical device) size of R plots in inches. R plots in code chunks are first recorded via a graphical device in \textbf{knitr}, and then written out to files. You can also specify the two options together in a single chunk option \texttt{fig.dim}, e.g., \texttt{fig.dim\ =\ c(6,\ 4)} means \texttt{fig.width\ =\ 6} and \texttt{fig.height\ =\ 4}.
\item
  \texttt{out.width} and \texttt{out.height}: The output size of R plots in the output document. These options may scale images. You can use percentages, e.g., \texttt{out.width\ =\ \textquotesingle{}80\%\textquotesingle{}} means 80\% of the page width.
\item
  \texttt{fig.align}: The alignment of plots. It can be \texttt{\textquotesingle{}left\textquotesingle{}}, \texttt{\textquotesingle{}center\textquotesingle{}}, or \texttt{\textquotesingle{}right\textquotesingle{}}.
\item
  \texttt{dev}: The graphical device to record R plots. Typically it is \texttt{\textquotesingle{}pdf\textquotesingle{}} for LaTeX output, and \texttt{\textquotesingle{}png\textquotesingle{}} for HTML output, but you can certainly use other devices, such as \texttt{\textquotesingle{}svg\textquotesingle{}} or \texttt{\textquotesingle{}jpeg\textquotesingle{}}.
\item
  \texttt{fig.cap}: The figure caption.
\item
  \texttt{child}: You can include a child document in the main document. This option takes a path to an external file.
\end{itemize}

\hypertarget{linguagens-suportadas}{%
\section{Linguagens suportadas}\label{linguagens-suportadas}}

\begin{Shaded}
\begin{Highlighting}[]
\KeywordTok{names}\NormalTok{(knitr}\OperatorTok{::}\NormalTok{knit\_engines}\OperatorTok{$}\KeywordTok{get}\NormalTok{())}
\end{Highlighting}
\end{Shaded}

\begin{verbatim}
##  [1] "awk"         "bash"        "coffee"      "gawk"        "groovy"     
##  [6] "haskell"     "lein"        "mysql"       "node"        "octave"     
## [11] "perl"        "psql"        "Rscript"     "ruby"        "sas"        
## [16] "scala"       "sed"         "sh"          "stata"       "zsh"        
## [21] "highlight"   "Rcpp"        "tikz"        "dot"         "c"          
## [26] "fortran"     "fortran95"   "asy"         "cat"         "asis"       
## [31] "stan"        "block"       "block2"      "js"          "css"        
## [36] "sql"         "go"          "python"      "julia"       "sass"       
## [41] "scss"        "theorem"     "lemma"       "corollary"   "proposition"
## [46] "conjecture"  "definition"  "example"     "exercise"    "proof"      
## [51] "remark"      "solution"
\end{verbatim}

citacoes \textbf{bookdown} {[}@R-bookdown{]}

\hypertarget{exemplo}{%
\section{Exemplo}\label{exemplo}}

\href{exemplos/104-intro_blocos_de_codigo.html}{Exemplo}

\hypertarget{tipos-adicionais-de-output}{%
\chapter{Tipos adicionais de Output}\label{tipos-adicionais-de-output}}

\hypertarget{apresentauxe7uxe3o-em-slides}{%
\section{Apresentação em Slides}\label{apresentauxe7uxe3o-em-slides}}

\hypertarget{pacotes-extras}{%
\section{Pacotes extras}\label{pacotes-extras}}

\hypertarget{referuxeancia}{%
\chapter{Referência}\label{referuxeancia}}

\textless!DOCTYPE html\textgreater{}

105-referencia.utf8

\hypertarget{header}{}

Sintaxe

~

Resultado

Texto

~

Texto

Termine uma linha com dois espaços para
começar um novo parágrafo.

~

Termine uma linha com dois espaços para começar um novo parágrafo.

\emph{Itálico} e \emph{itálico}

~

Itálico e itálico

\textbf{Negrito} e \textbf{negrito}

~

Negrito e negrito

Sobrescrito\textsuperscript{2}

~

Sobrescrito2

\sout{Texto taxado}

~

Texto taxado

\href{des.uem.br}{Link}

~

Link

\# Cabeçalho 1

~

\hypertarget{cabeuxe7alho-1}{}
Cabeçalho 1

\#\# Cabeçalho 2

~

\hypertarget{cabeuxe7alho-2}{}
Cabeçalho 2

\#\#\# Cabeçalho 3

~

\hypertarget{cabeuxe7alho-3}{}
Cabeçalho 3

\#\#\#\# Cabeçalho 4

~

\hypertarget{cabeuxe7alho-4}{}
Cabeçalho 4

\#\#\#\#\# Cabeçalho 5

~

\hypertarget{cabeuxe7alho-5}{}
Cabeçalho 5

\#\#\#\#\#\# Cabeçalho 6

~

\hypertarget{cabeuxe7alho-6}{}
Cabeçalho 6

Expressão matemática:
\(f_x(x) = \lambda e^{- \lambda x}\)

~

Expressão matemática: {\(f_x(x) = \lambda e^{- \lambda x}\)}

Expressão matemática em bloco:

\[X = \begin{bmatrix}1 &amp; x_{1}\\</code><br />
<code>1 &amp; x_{2}\\</code><br />
<code>1 &amp; x_{3}</code><br />
<code>\end{bmatrix}\]

~

Expressão matemática em bloco:

{\[X = \begin{bmatrix}1 &amp; x_{1}\\
1 &amp; x_{2}\\
1 &amp; x_{3}
\end{bmatrix}\]}

Imagem: \includegraphics{http://tny.im/lTZ}

~

Imagem:

Linha horizontal:

***

~

Linha horizontal:

Exemplo de nota de rodapé\footnote{Nota de rodapé}

~

Exemplo de nota de rodapé1

~

item

item

item

item

item

item

item

item

item

~

item 1

item 2

item 3

item 1

item 2

~

``É notável uma ciência que começou com jogos de azar tenha se tornado o mais importante objeto do conhecimento humano.''

--- Pierre Simon Laplace

~

~

Coluna 1

Coluna 2

11

12

21

22

Rodando código \texttt{r\ “no\ meio\ do\ texto”}

~

Rodando código no meio do texto

~

~

~

~

Cars

speed

dist

4

2

4

10

7

4

7

22

8

16

~

Nota de rodapé↩︎

\end{document}
