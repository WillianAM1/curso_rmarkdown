% Options for packages loaded elsewhere
\PassOptionsToPackage{unicode}{hyperref}
\PassOptionsToPackage{hyphens}{url}
%
\documentclass[
]{book}
\usepackage{lmodern}
\usepackage{amssymb,amsmath}
\usepackage{ifxetex,ifluatex}
\ifnum 0\ifxetex 1\fi\ifluatex 1\fi=0 % if pdftex
  \usepackage[T1]{fontenc}
  \usepackage[utf8]{inputenc}
  \usepackage{textcomp} % provide euro and other symbols
\else % if luatex or xetex
  \usepackage{unicode-math}
  \defaultfontfeatures{Scale=MatchLowercase}
  \defaultfontfeatures[\rmfamily]{Ligatures=TeX,Scale=1}
\fi
% Use upquote if available, for straight quotes in verbatim environments
\IfFileExists{upquote.sty}{\usepackage{upquote}}{}
\IfFileExists{microtype.sty}{% use microtype if available
  \usepackage[]{microtype}
  \UseMicrotypeSet[protrusion]{basicmath} % disable protrusion for tt fonts
}{}
\makeatletter
\@ifundefined{KOMAClassName}{% if non-KOMA class
  \IfFileExists{parskip.sty}{%
    \usepackage{parskip}
  }{% else
    \setlength{\parindent}{0pt}
    \setlength{\parskip}{6pt plus 2pt minus 1pt}}
}{% if KOMA class
  \KOMAoptions{parskip=half}}
\makeatother
\usepackage{xcolor}
\IfFileExists{xurl.sty}{\usepackage{xurl}}{} % add URL line breaks if available
\IfFileExists{bookmark.sty}{\usepackage{bookmark}}{\usepackage{hyperref}}
\hypersetup{
  pdftitle={Introdução ao R Markdown},
  pdfauthor={Eduardo José de Campos Lemos Júnior, Samuel Vianna Quintanilha},
  hidelinks,
  pdfcreator={LaTeX via pandoc}}
\urlstyle{same} % disable monospaced font for URLs
\usepackage{color}
\usepackage{fancyvrb}
\newcommand{\VerbBar}{|}
\newcommand{\VERB}{\Verb[commandchars=\\\{\}]}
\DefineVerbatimEnvironment{Highlighting}{Verbatim}{commandchars=\\\{\}}
% Add ',fontsize=\small' for more characters per line
\usepackage{framed}
\definecolor{shadecolor}{RGB}{248,248,248}
\newenvironment{Shaded}{\begin{snugshade}}{\end{snugshade}}
\newcommand{\AlertTok}[1]{\textcolor[rgb]{0.94,0.16,0.16}{#1}}
\newcommand{\AnnotationTok}[1]{\textcolor[rgb]{0.56,0.35,0.01}{\textbf{\textit{#1}}}}
\newcommand{\AttributeTok}[1]{\textcolor[rgb]{0.77,0.63,0.00}{#1}}
\newcommand{\BaseNTok}[1]{\textcolor[rgb]{0.00,0.00,0.81}{#1}}
\newcommand{\BuiltInTok}[1]{#1}
\newcommand{\CharTok}[1]{\textcolor[rgb]{0.31,0.60,0.02}{#1}}
\newcommand{\CommentTok}[1]{\textcolor[rgb]{0.56,0.35,0.01}{\textit{#1}}}
\newcommand{\CommentVarTok}[1]{\textcolor[rgb]{0.56,0.35,0.01}{\textbf{\textit{#1}}}}
\newcommand{\ConstantTok}[1]{\textcolor[rgb]{0.00,0.00,0.00}{#1}}
\newcommand{\ControlFlowTok}[1]{\textcolor[rgb]{0.13,0.29,0.53}{\textbf{#1}}}
\newcommand{\DataTypeTok}[1]{\textcolor[rgb]{0.13,0.29,0.53}{#1}}
\newcommand{\DecValTok}[1]{\textcolor[rgb]{0.00,0.00,0.81}{#1}}
\newcommand{\DocumentationTok}[1]{\textcolor[rgb]{0.56,0.35,0.01}{\textbf{\textit{#1}}}}
\newcommand{\ErrorTok}[1]{\textcolor[rgb]{0.64,0.00,0.00}{\textbf{#1}}}
\newcommand{\ExtensionTok}[1]{#1}
\newcommand{\FloatTok}[1]{\textcolor[rgb]{0.00,0.00,0.81}{#1}}
\newcommand{\FunctionTok}[1]{\textcolor[rgb]{0.00,0.00,0.00}{#1}}
\newcommand{\ImportTok}[1]{#1}
\newcommand{\InformationTok}[1]{\textcolor[rgb]{0.56,0.35,0.01}{\textbf{\textit{#1}}}}
\newcommand{\KeywordTok}[1]{\textcolor[rgb]{0.13,0.29,0.53}{\textbf{#1}}}
\newcommand{\NormalTok}[1]{#1}
\newcommand{\OperatorTok}[1]{\textcolor[rgb]{0.81,0.36,0.00}{\textbf{#1}}}
\newcommand{\OtherTok}[1]{\textcolor[rgb]{0.56,0.35,0.01}{#1}}
\newcommand{\PreprocessorTok}[1]{\textcolor[rgb]{0.56,0.35,0.01}{\textit{#1}}}
\newcommand{\RegionMarkerTok}[1]{#1}
\newcommand{\SpecialCharTok}[1]{\textcolor[rgb]{0.00,0.00,0.00}{#1}}
\newcommand{\SpecialStringTok}[1]{\textcolor[rgb]{0.31,0.60,0.02}{#1}}
\newcommand{\StringTok}[1]{\textcolor[rgb]{0.31,0.60,0.02}{#1}}
\newcommand{\VariableTok}[1]{\textcolor[rgb]{0.00,0.00,0.00}{#1}}
\newcommand{\VerbatimStringTok}[1]{\textcolor[rgb]{0.31,0.60,0.02}{#1}}
\newcommand{\WarningTok}[1]{\textcolor[rgb]{0.56,0.35,0.01}{\textbf{\textit{#1}}}}
\usepackage{longtable,booktabs}
% Correct order of tables after \paragraph or \subparagraph
\usepackage{etoolbox}
\makeatletter
\patchcmd\longtable{\par}{\if@noskipsec\mbox{}\fi\par}{}{}
\makeatother
% Allow footnotes in longtable head/foot
\IfFileExists{footnotehyper.sty}{\usepackage{footnotehyper}}{\usepackage{footnote}}
\makesavenoteenv{longtable}
\usepackage{graphicx}
\makeatletter
\def\maxwidth{\ifdim\Gin@nat@width>\linewidth\linewidth\else\Gin@nat@width\fi}
\def\maxheight{\ifdim\Gin@nat@height>\textheight\textheight\else\Gin@nat@height\fi}
\makeatother
% Scale images if necessary, so that they will not overflow the page
% margins by default, and it is still possible to overwrite the defaults
% using explicit options in \includegraphics[width, height, ...]{}
\setkeys{Gin}{width=\maxwidth,height=\maxheight,keepaspectratio}
% Set default figure placement to htbp
\makeatletter
\def\fps@figure{htbp}
\makeatother
\usepackage[normalem]{ulem}
% Avoid problems with \sout in headers with hyperref
\pdfstringdefDisableCommands{\renewcommand{\sout}{}}
\setlength{\emergencystretch}{3em} % prevent overfull lines
\providecommand{\tightlist}{%
  \setlength{\itemsep}{0pt}\setlength{\parskip}{0pt}}
\setcounter{secnumdepth}{5}

\title{Introdução ao R Markdown}
\author{Eduardo José de Campos Lemos Júnior, Samuel Vianna Quintanilha}
\date{2020-08-25}

\begin{document}
\maketitle

{
\setcounter{tocdepth}{1}
\tableofcontents
}
\hypertarget{intro}{%
\chapter{Introdução}\label{intro}}

Neste curso, iremos aprender a utilizar o R Markdown para gerar documentos.

\hypertarget{o-que-uxe9}{%
\section{O que é}\label{o-que-uxe9}}

Markdown é uma linguagem de marcação usada para formatar de maneira simples os textos redigidos e converte-los em HTML. John Gruber e Aaron Swartz, os criadores desse sistema, utilizaram marcadores como: \texttt{\#,\ \textbackslash{}*\ ,!\ ,\ {[}{]}\ e\ ()}. Assim é possível inserir em nossos textos: títulos, listas, formatação de fonte, imagens e tabelas.
R Markdown é um documento criado no R Studio (Ou em outras IDEs) que possibilita empregar os recursos da linguagem markdown citados acima em conjunto com a linguagem R, permitindo a melhor organização de análises, relatórios e códigos em um só documento.

\hypertarget{possuxedveis-tipos-de-outputs}{%
\section{Possíveis tipos de Outputs}\label{possuxedveis-tipos-de-outputs}}

O R Markdown apresenta várias possibilidades de outputs (renderizar) nos formatos de documentos, apresentações, entre outros, sendo que em cada formato há várias opções de customização. Vejamos abaixo os principais:

Documentos:

\begin{itemize}
\tightlist
\item
  html\_document -- documento no formato HTML;
\item
  pdf\_document -- documento no formato PDF (via o modelo LaTeX);
\item
  word\_document -- documento no formato do editor de texto Microsoft Word (docx);
\item
  odt\_documento -- documento no formato dos editores de texto Libre Office e OpenDocument;
\item
  rtf\_documento -- documento no formato Rich Text Format (rtf).
\end{itemize}

Apresentações (slides):

\begin{itemize}
\tightlist
\item
  ioslides\_presentation -- apresentação no formato HTML com ioslides;
\item
  beamer\_presentation -- apresentação no formato PDF com LaTeX Beamer;
\item
  powerpoint\_presentation -- apresentação no formato power point.
\end{itemize}

Outros:

\begin{itemize}
\tightlist
\item
  flexdashboard::flex\_dashboard -- apresentação interativa com dashboards;
\item
  htm\_vignette -- R package vignette no format HTML
\item
  github\_document -- document no format GitHub
\end{itemize}

Você pode escolher o output desejado quando for criar um documento conforme a figura 1. Para fazer deve-se clicar em \texttt{file\ \textgreater{}\ new\ file\ \textgreater{}\ R\ Markdown}. Abrirá uma aba e nela há quatro formas de output previamente estabelecidas, a saber: Document (HTML, PDF e Word), Presentation (HTML (ioslides), HTML (slidy), PDF (Beamer) e PowerPoint), Shiny (Shiny Document e Shy Presentation) e From Template (GitHub document e Package Vignette). Escolha uma e clique em OK.
Além disso, você também pode alterar o formato utilizando a função abaixo, sendo que render refere-se ao local que está salvo seu documento e output\_format ao tipo de documento desejado, conforme os exemplos apontados no início.

\begin{Shaded}
\begin{Highlighting}[]
\KeywordTok{render}\NormalTok{(}\StringTok{"teste.Rmd"}\NormalTok{, }\DataTypeTok{output\_format =} \StringTok{"pdf\_document"}\NormalTok{)}
\end{Highlighting}
\end{Shaded}

O mesmo pode ser feito para outros formatos.
Abaixo está presente uma lista com todos os formatos suportados por padrão com o pacote \texttt{rmarkdown}.

\begin{itemize}
\tightlist
\item
  \texttt{beamer\_presentation}
\item
  \texttt{context\_document}
\item
  \texttt{github\_document}
\item
  \texttt{html\_document}
\item
  \texttt{ioslides\_presentation}
\item
  \texttt{latex\_document}
\item
  \texttt{md\_document}
\item
  \texttt{odt\_document}
\item
  \texttt{pdf\_document}
\item
  \texttt{powerpoint\_presentation}
\item
  \texttt{rtf\_document}
\item
  \texttt{slidy\_presentation}
\item
  \texttt{word\_document}
\end{itemize}

\hypertarget{criando-o-primeiro-documento}{%
\section{Criando o primeiro documento}\label{criando-o-primeiro-documento}}

Para gerar um arquivo em R Markdown é necessário abrir o programa R Studio, e instalar o pacote rmarkdown :

\begin{Shaded}
\begin{Highlighting}[]
\KeywordTok{install.packages}\NormalTok{(}\StringTok{"rmarkdown"}\NormalTok{)}
\end{Highlighting}
\end{Shaded}

Após a instalação do pacote no R Studio, siga os seguintes passos:
\includegraphics{img/new.png}

Em seguida, escolha o tipo de arquivo desejado:
\includegraphics{img/file.png}

Obs: Para gerar documentos em PDF, é necessário ter instalado em seu computador o programa \protect\hyperlink{download-rmd-e-latex}{Latex}

Seguindo os passos acima, você terá criado o seu primeiro documento em R Markdown.

Vale ressaltar que é possível utilizar o R Markdown sem que tenha instalado o R Studio, porém, é necessário ter instalado o programa \href{https://pandoc.org}{Pandoc}

\hypertarget{download-miktex}{%
\section{Download MiKTeX}\label{download-miktex}}

Para exportar um arquivo PDF utilizando o R Markdown é necessário um motor LaTex pois é ele que irá converter o arquivo .tex em PDF.
Então é necessário que tenha instalado em seu computador o programa MiKTeX,. Para fazer o download é só acessar o link: \url{https://miktex.org/download}.

\hypertarget{windows}{%
\subsection{Windows}\label{windows}}

Selecione a aba Windows e clique no botão de download:
\includegraphics{img/download_windowns_MiKTex.jpg}

\hypertarget{linux}{%
\subsection{Linux}\label{linux}}

Selecione a aba Linux em seguida a aba de sua distribuição Linux para receber as instruções de instalação:
\includegraphics{img/download_linux_MiKTex.jpg}

\hypertarget{macos}{%
\subsection{MacOS}\label{macos}}

Selecione a aba macOS e clique no botão de download:
\includegraphics{img/download_macOS_MiKTex.jpg}

\hypertarget{sintaxe}{%
\chapter{Sintaxe}\label{sintaxe}}

\hypertarget{pruxeaambulo}{%
\section{Prêambulo}\label{pruxeaambulo}}

No início de um documento R Markdown, é utilizada a linguagem \href{https://pt.wikipedia.org/wiki/YAML}{yaml} para definir as configurações do seu arquivo.
As configurações disponiveis no preâmbulo do seu documento R Markdown são variadas, podendo inclusive ter diferentes opções para diferentes tipos de arquivos.
Uma das configurações mais importantes para se definir no seu preambulo é o \protect\hyperlink{possuxedveis-tipos-de-outputs}{tipo de documento} a ser gerado.

\hypertarget{definindo-tipo-de-output}{%
\subsection{Definindo tipo de output}\label{definindo-tipo-de-output}}

A opçao de \protect\hyperlink{possuxedveis-tipos-de-outputs}{tipo de documento} é definida da seguinte forma:

\begin{verbatim}
---
output: pdf_document
---
\end{verbatim}

No código acima, \texttt{output} foi definido como \texttt{pdf\_document()}. Output é o formato final do seu documento, e \texttt{pdf\_document()} é, dentro do pacote \texttt{rmarkdown}, o formato pdf. Para gerar arquivos de formatos diferentes, é necessário somente que seja modificado a opção de \texttt{output} no preâmbulo.
Note que a opção definida no preâmbulo foi escrita entre duas linhas tracejadas, não é possível definir suas configurações fora dessas linhas tracejadas, da mesma forma, não é possível escrever partes do seu documento ou rodar códigos de \texttt{R} dentro das linhas.
Por padrão no \texttt{yaml} utilizado no preâmbulo de R Markdown, a opção a ser definida é escrita sem espaços, seguida de dois pontos, espaço e então a definição da opção.
Existem opções dentro de outras opções, como por exemplo \texttt{pdf\_document()}, podemos definir configurações se o documento terá ou não sumário, quantos níveis de \protect\hyperlink{cabeuxe7alhos}{cabeçalho} serão utilizados no sumário, etc.

\begin{verbatim}
---
output: pdf_document:
  toc: true
  toc_depth: 3
  latex_engine: xelatex
---
\end{verbatim}

No exemplo acima, definimos \texttt{toc} (table of contents) como \texttt{true}, o que vai fazer com que seja gerado um sumário no documento final, perceba que em \texttt{yaml} o valor lógico de verdadeiro é escrito com todas as letras minúsculas. Definindo \texttt{toc\_depth} como \texttt{3}, quando o sumário for gerado, até três níveis de \protect\hyperlink{cabeuxe7alhos}{cabeçalho} apareceram no sumário.

\hypertarget{informauxe7uxf5es-gerais}{%
\subsection{Informações Gerais}\label{informauxe7uxf5es-gerais}}

Além do tipo de documento, e das opções de cada tipo de documento, podemos definir algumas opções gerais, como autor, título do documento, e data.

\begin{verbatim}
---
author: "Fulano"
output: pdf_document:
  toc: true
  toc_depth: 3
  latex_engine: xelatex
---
\end{verbatim}

Seguindo com o preâmbulo já feito anteriormente, foi adicionada a opção \texttt{author}, que irá definir o autor do seu documento. Por padrão, o autor irá aparecer na página inicial de diversos \protect\hyperlink{possuxedveis-tipos-de-outputs}{tipos de documento}.

\begin{verbatim}
---
date: 1 de janeiro de 1970
output: pdf_document:
  toc: true
  toc_depth: 3
  latex_engine: xelatex
author: Fulano
---
\end{verbatim}

Utilizando a opção \texttt{date} podemos definir uma data para o documento, semelhante ao autor, a data aparece por padrão no início dos documentos.
Perceba que no exemplo acima, o nome do autor e a data foram escritas sem aspas, mas irão funcionar da mesma forma, perceba também que a opção do autor foi trocada de ordem, as opções do preâmbulo não necessitam de uma ordem específica, mas as opções de dentro de outras opções devem sempre estar abaixo da opção mae (sei la como chamar isso) e com uma identação a mais.

\begin{verbatim}
---
author: Fulano
date: 1 de janeiro de 1970
title: Título
output: pdf_document:
  toc: true
  toc_depth: 3
  latex_engine: xelatex
---
\end{verbatim}

Por fim foi definida também a opção \texttt{title} que irá definir o título do documento final. Para maior customização do seu documento pelo preâmbulo veja o \protect\hyperlink{depois-crio}{capítulo 4}.

\hypertarget{sumario}{%
\subsection{Sumario}\label{sumario}}

\hypertarget{textos}{%
\section{Textos}\label{textos}}

\hypertarget{tuxedtulos}{%
\subsection{Títulos}\label{tuxedtulos}}

\begin{verbatim}
# Título Nível 1

## Título Nível 2

### Título Nível 3

#### Título Nível 4

##### Título Nível 5

###### Título Nível 6
\end{verbatim}

\hypertarget{formatauxe7uxe3o-de-textos-negrito-ituxe1lico-sobrescrito-tachado-e-cuxf3digo}{%
\subsection{Formatação de textos (negrito, itálico, sobrescrito, tachado e código)}\label{formatauxe7uxe3o-de-textos-negrito-ituxe1lico-sobrescrito-tachado-e-cuxf3digo}}

\begin{verbatim}
*Negrito* _Negrito_
\end{verbatim}

\textbf{Negrito} \textbf{Negrito}

\begin{verbatim}
**Italico** __italico__
\end{verbatim}

\emph{Italico} \emph{Italico}

\begin{verbatim}
texto^sobrescrito^
\end{verbatim}

texto\textsuperscript{sobrescrito}

\begin{verbatim}
~~~tachado~~
\end{verbatim}

\textasciitilde{}\sout{tachado}

\hypertarget{links}{%
\subsection{Links}\label{links}}

Para um link devemos utilizar a seguinte sintaxe:

\begin{verbatim}
[nome do link](url do link)
[Curso de RMarkdown](https://estatsej.github.io/curso_rmarkdown)
\end{verbatim}

Existem também outras variações para que utlizemos os links em nosso material.

\textbf{1 - Nome do link seja ele próprio}:

\url{https://estatsej.github.io/curso_rmarkdown}

\textbf{2- Link contendo um título, que aparece ao deixar o cursor do mouse em cima do link:}

\begin{verbatim}
[Curso de RMakrdown](https://estatsej.github.io/curso_rmarkdown "Aqui está o nosso curso de RMarkdown")
\end{verbatim}

Podemos observar:
\href{https://estatsej.github.io/curso_rmarkdown}{Curso de RMakrdown}

\hypertarget{listas}{%
\subsection{Listas}\label{listas}}

\textbf{Lista Ordenada}

\begin{verbatim}
1. Primeiro item
2. Segundo item
3. Terceiro item
\end{verbatim}

\textbf{Lista Não-Ordenada}

\begin{verbatim}
- Primeiro item
- Segundo item
- Terceiro item
\end{verbatim}

\textbf{Lista com Sublista}

\begin{verbatim}
1. Item
    - Um sub-item
    - Outro sub-item
\end{verbatim}

\hypertarget{inserindo-imagens}{%
\section{Inserindo Imagens}\label{inserindo-imagens}}

Antes de inserir a imagem escolhida podemos definir a configuração global para todas as imagens, lembrando que a configuração feita diretamente na imagem vai sobrepor a configuração global.

\begin{Shaded}
\begin{Highlighting}[]
\InformationTok{\textasciigrave{}\textasciigrave{}\textasciigrave{}\{r setup, include=FALSE\}}
\InformationTok{library(knitr)}
\InformationTok{opts\_chunk$set(echo = FALSE,}
\InformationTok{               out.width = "10\%", }
\InformationTok{               fig.align = "center")}
\InformationTok{\textasciigrave{}\textasciigrave{}\textasciigrave{}}
\end{Highlighting}
\end{Shaded}

Se um gráfico ou imagem não for gerado a partir de um código feito por R, você poderá incluí-lo de duas maneiras:

\begin{itemize}
\tightlist
\item
  Usando a sintaxe Markdown \texttt{!{[}texto{]}(pasta/para/image)}. Nesse caso, você pode definir o tamanho da imagem usando os atributos \texttt{width} ou \texttt{height}, por exemplo:
\end{itemize}

\begin{verbatim}
![Logo da Estats](img/logoestats.jpeg){width=50%}
\end{verbatim}

\begin{itemize}
\tightlist
\item
  Usando a função \texttt{knitr::include\_graphics(}) em qualquer parte do código. Você pode usar opções de configuração de tamanho como \texttt{out.width} ou \texttt{out.height} para esse exemplo:
\end{itemize}

\begin{Shaded}
\begin{Highlighting}[]
\InformationTok{\textasciigrave{}\textasciigrave{}\textasciigrave{}\{r, echo=FALSE, out.width="50\%", fig.cap="Logo da Estats"\}}
\InformationTok{knitr::include\_graphics("img/logoestats.jpeg")}
\InformationTok{\textasciigrave{}\textasciigrave{}\textasciigrave{}}
\end{Highlighting}
\end{Shaded}

Se você souber que deseja gerar a imagem apenas para um formato de saída específico, poderá usar uma unidade específica. Por exemplo, você pode usar \texttt{out.width="300px"} se o formato de saída for HTML, mas no nosso exemplo o formato que usamos \texttt{out.width="50\%"} é válido para qualquer saída.

E podemos aliar as imagens utilizando \texttt{fig.align}. Por exemplo, você pode centralizar imagens \texttt{fig.align="center"} ou alinhar à direita as imagens \texttt{fig.align="right\textquotesingle{}"}. Está opção funciona para saída HTML e LaTeX, mas pode não funcionar para outros formatos de saída (como o Word).

No código a seguir é utilizado quando o seu projeto tem múltiplas saídas (PDF,HTML,\ldots) e ocorra problema na inclusão das imagens, então a sugestão para o melhor caminho foi fazer a validação para obter a saída utilizada, sabendo disso podemos fazer configuração e inserindo imagens especifica para cada saída.

\begin{Shaded}
\begin{Highlighting}[]
\InformationTok{\textasciigrave{}\textasciigrave{}\textasciigrave{}\{r, fig.cap="Logo da Estats"\}}
\InformationTok{if (knitr::is\_html\_output()) \{}
\InformationTok{    knitr::include\_graphics("img/logoestats.jpeg")}
\InformationTok{\} else \{}
\InformationTok{    knitr::include\_graphics("img/logoestats.jpeg")}
\InformationTok{\}}
\InformationTok{\textasciigrave{}\textasciigrave{}\textasciigrave{}}
\end{Highlighting}
\end{Shaded}

\begin{Shaded}
\begin{Highlighting}[]
\ControlFlowTok{if}\NormalTok{ (knitr}\OperatorTok{::}\KeywordTok{is\_html\_output}\NormalTok{()) \{}
\NormalTok{    knitr}\OperatorTok{::}\KeywordTok{include\_graphics}\NormalTok{(}\StringTok{"img/logoestats.jpeg"}\NormalTok{)}
\NormalTok{\} }\ControlFlowTok{else}\NormalTok{ \{}
\NormalTok{    knitr}\OperatorTok{::}\KeywordTok{include\_graphics}\NormalTok{(}\StringTok{"img/logoestats.jpeg"}\NormalTok{)}
\NormalTok{\}}
\end{Highlighting}
\end{Shaded}

\begin{figure}

{\centering \includegraphics[width=0.89in]{img/logoestats} 

}

\caption{Logo da Estats}\label{fig:unnamed-chunk-3}
\end{figure}

Os dois exemplos abaixo são casos de curiosidades:

\begin{itemize}
\tightlist
\item
  Nesse caso podemos utilizar vetor para colocar várias imagem juntas.
\end{itemize}

\begin{Shaded}
\begin{Highlighting}[]
\InformationTok{\textasciigrave{}\textasciigrave{}\textasciigrave{}\{r image\}}
\InformationTok{include\_graphics(c("img1.jpeg", "img2.jpeg"))}
\InformationTok{\textasciigrave{}\textasciigrave{}\textasciigrave{}}
\end{Highlighting}
\end{Shaded}

\begin{itemize}
\tightlist
\item
  E nesse último caso podemos fazer repetição de uma imagem várias vezes.
\end{itemize}

\begin{Shaded}
\begin{Highlighting}[]
\InformationTok{\textasciigrave{}\textasciigrave{}\textasciigrave{}\{r\}}
\InformationTok{knitr::include\_graphics(rep("img/logoestats.jpeg", 3))}
\InformationTok{\textasciigrave{}\textasciigrave{}\textasciigrave{}}
\end{Highlighting}
\end{Shaded}

\begin{Shaded}
\begin{Highlighting}[]
\NormalTok{knitr}\OperatorTok{::}\KeywordTok{include\_graphics}\NormalTok{(}\KeywordTok{rep}\NormalTok{(}\StringTok{"img/logoestats.jpeg"}\NormalTok{, }\DecValTok{3}\NormalTok{))}
\end{Highlighting}
\end{Shaded}

\includegraphics[width=0.89in]{img/logoestats}
\includegraphics[width=0.89in]{img/logoestats}
\includegraphics[width=0.89in]{img/logoestats}

\hypertarget{exemplos}{%
\section{Exemplos}\label{exemplos}}

\href{exemplos/102-sumario_e_cabecalho.html}{Exemplo Sumário e Cabeçalhos}

\href{exemplos/103-formatacao_de_texto.html}{Exemplo formatação de textos}

\hypertarget{executando-cuxf3digo}{%
\chapter{Executando Código}\label{executando-cuxf3digo}}

\hypertarget{introduuxe7uxe3o}{%
\section{Introdução}\label{introduuxe7uxe3o}}

\begin{verbatim}
```{r}

```
\end{verbatim}

\hypertarget{flags}{%
\section{Flags}\label{flags}}

There are a large number of chunk options\index{chunk options} in \textbf{knitr} documented at \url{https://yihui.name/knitr/options}. We list a subset of them below:

\begin{itemize}
\item
  \texttt{eval}: Whether to evaluate a code chunk.
\item
  \texttt{echo}: Whether to echo the source code in the output document (someone may not prefer reading your smart source code but only results).
\item
  \texttt{results}: When set to \texttt{\textquotesingle{}hide\textquotesingle{}}, text output will be hidden; when set to \texttt{\textquotesingle{}asis\textquotesingle{}}, text output is written ``as-is'', e.g., you can write out raw Markdown text from R code (like \texttt{cat(\textquotesingle{}**Markdown**\ is\ cool.\textbackslash{}n\textquotesingle{})}). By default, text output will be wrapped in verbatim elements (typically plain code blocks).
\item
  \texttt{collapse}: Whether to merge text output and source code into a single code block in the output. This is mostly cosmetic: \texttt{collapse\ =\ TRUE} makes the output more compact, since the R source code and its text output are displayed in a single output block. The default \texttt{collapse\ =\ FALSE} means R expressions and their text output are separated into different blocks.
\item
  \texttt{warning}, \texttt{message}, and \texttt{error}: Whether to show warnings, messages, and errors in the output document. Note that if you set \texttt{error\ =\ FALSE}, \texttt{rmarkdown::render()} will halt on error in a code chunk, and the error will be displayed in the R console. Similarly, when \texttt{warning\ =\ FALSE} or \texttt{message\ =\ FALSE}, these messages will be shown in the R console.
\item
  \texttt{include}: Whether to include anything from a code chunk in the output document. When \texttt{include\ =\ FALSE}, this whole code chunk is excluded in the output, but note that it will still be evaluated if \texttt{eval\ =\ TRUE}. When you are trying to set \texttt{echo\ =\ FALSE}, \texttt{results\ =\ \textquotesingle{}hide\textquotesingle{}}, \texttt{warning\ =\ FALSE}, and \texttt{message\ =\ FALSE}, chances are you simply mean a single option \texttt{include\ =\ FALSE} instead of suppressing different types of text output individually.
\item
  \texttt{cache}: Whether to enable caching. If caching is enabled, the same code chunk will not be evaluated the next time the document is compiled (if the code chunk was not modified), which can save you time. However, I want to honestly remind you of the two hard problems in computer science (via Phil Karlton): naming things, and cache invalidation. Caching can be handy but also tricky sometimes.
\item
  \texttt{fig.width} and \texttt{fig.height}: The (graphical device) size of R plots in inches. R plots in code chunks are first recorded via a graphical device in \textbf{knitr}, and then written out to files. You can also specify the two options together in a single chunk option \texttt{fig.dim}, e.g., \texttt{fig.dim\ =\ c(6,\ 4)} means \texttt{fig.width\ =\ 6} and \texttt{fig.height\ =\ 4}.
\item
  \texttt{out.width} and \texttt{out.height}: The output size of R plots in the output document. These options may scale images. You can use percentages, e.g., \texttt{out.width\ =\ \textquotesingle{}80\%\textquotesingle{}} means 80\% of the page width.
\item
  \texttt{fig.align}: The alignment of plots. It can be \texttt{\textquotesingle{}left\textquotesingle{}}, \texttt{\textquotesingle{}center\textquotesingle{}}, or \texttt{\textquotesingle{}right\textquotesingle{}}.
\item
  \texttt{dev}: The graphical device to record R plots. Typically it is \texttt{\textquotesingle{}pdf\textquotesingle{}} for LaTeX output, and \texttt{\textquotesingle{}png\textquotesingle{}} for HTML output, but you can certainly use other devices, such as \texttt{\textquotesingle{}svg\textquotesingle{}} or \texttt{\textquotesingle{}jpeg\textquotesingle{}}.
\item
  \texttt{fig.cap}: The figure caption.
\item
  \texttt{child}: You can include a child document in the main document. This option takes a path to an external file.
\end{itemize}

\hypertarget{linguagens-suportadas}{%
\section{Linguagens suportadas}\label{linguagens-suportadas}}

\begin{Shaded}
\begin{Highlighting}[]
\KeywordTok{names}\NormalTok{(knitr}\OperatorTok{::}\NormalTok{knit\_engines}\OperatorTok{$}\KeywordTok{get}\NormalTok{())}
\end{Highlighting}
\end{Shaded}

\begin{verbatim}
##  [1] "awk"         "bash"        "coffee"      "gawk"        "groovy"     
##  [6] "haskell"     "lein"        "mysql"       "node"        "octave"     
## [11] "perl"        "psql"        "Rscript"     "ruby"        "sas"        
## [16] "scala"       "sed"         "sh"          "stata"       "zsh"        
## [21] "highlight"   "Rcpp"        "tikz"        "dot"         "c"          
## [26] "cc"          "fortran"     "fortran95"   "asy"         "cat"        
## [31] "asis"        "stan"        "block"       "block2"      "js"         
## [36] "css"         "sql"         "go"          "python"      "julia"      
## [41] "sass"        "scss"        "theorem"     "lemma"       "corollary"  
## [46] "proposition" "conjecture"  "definition"  "example"     "exercise"   
## [51] "proof"       "remark"      "solution"
\end{verbatim}

\hypertarget{exemplo}{%
\section{Exemplo}\label{exemplo}}

Para que o código seja executado no meio de um texto, é necessário usar o caractere \texttt{\textasciigrave{}}, antes e depois do código descrito. Exemplo:

\begin{Shaded}
\begin{Highlighting}[]
\NormalTok{Este texto gera}\OperatorTok{:}\StringTok{ \textasciigrave{}}\DataTypeTok{r format(Sys.Date(), "\%d/\%m/\%Y")}\StringTok{\textasciigrave{}}\NormalTok{.}
\end{Highlighting}
\end{Shaded}

Este texto gera: 25/08/2020.

\begin{center}\rule{0.5\linewidth}{0.5pt}\end{center}

O código também pode ser executado dentro de blocos. Blocos de código são inseridos da seguinte forma:

\begin{Shaded}
\begin{Highlighting}[]
\InformationTok{\textasciigrave{}\textasciigrave{}\textasciigrave{}\{r\}}
\InformationTok{format(Sys.Date(), "\%d/\%m/\%Y")}
\InformationTok{\textasciigrave{}\textasciigrave{}\textasciigrave{}}
\end{Highlighting}
\end{Shaded}

O que gera o seguinte output ao ser compilado seu arquivo R Markdown:

\begin{Shaded}
\begin{Highlighting}[]
\KeywordTok{format}\NormalTok{(}\KeywordTok{Sys.Date}\NormalTok{(), }\StringTok{"\%d/\%m/\%Y"}\NormalTok{)}
\end{Highlighting}
\end{Shaded}

\begin{verbatim}
## [1] "25/08/2020"
\end{verbatim}

\begin{center}\rule{0.5\linewidth}{0.5pt}\end{center}

Algumas das opções básicas de blocos de código \protect\hyperlink{flags}{já mencionadas}:

\begin{Shaded}
\begin{Highlighting}[]
\InformationTok{\textasciigrave{}\textasciigrave{}\textasciigrave{}\{r, echo = FALSE\}}
\InformationTok{2 + 2}
\InformationTok{\textasciigrave{}\textasciigrave{}\textasciigrave{}}
\end{Highlighting}
\end{Shaded}

\begin{verbatim}
## [1] 4
\end{verbatim}

\begin{Shaded}
\begin{Highlighting}[]
\InformationTok{\textasciigrave{}\textasciigrave{}\textasciigrave{}\{r, eval = FALSE\}}
\InformationTok{2 + 2}
\InformationTok{\textasciigrave{}\textasciigrave{}\textasciigrave{}}
\end{Highlighting}
\end{Shaded}

\begin{Shaded}
\begin{Highlighting}[]
\DecValTok{2} \OperatorTok{+}\StringTok{ }\DecValTok{2}
\end{Highlighting}
\end{Shaded}

\begin{center}\rule{0.5\linewidth}{0.5pt}\end{center}

Exemplo de utilização de blocos de código para inserir imagens como descrito no \protect\hyperlink{inserindo-imagens}{capítulo 2}

\begin{Shaded}
\begin{Highlighting}[]
\InformationTok{\textasciigrave{}\textasciigrave{}\textasciigrave{}\{r\}}
\InformationTok{knitr::include\_graphics("img/logoestats.jpeg")}
\InformationTok{\textasciigrave{}\textasciigrave{}\textasciigrave{}}
\end{Highlighting}
\end{Shaded}

\begin{Shaded}
\begin{Highlighting}[]
\NormalTok{knitr}\OperatorTok{::}\KeywordTok{include\_graphics}\NormalTok{(}\StringTok{"img/logoestats.jpeg"}\NormalTok{)}
\end{Highlighting}
\end{Shaded}

\includegraphics[width=0.89in]{img/logoestats}

\begin{center}\rule{0.5\linewidth}{0.5pt}\end{center}

Inserindo tabelas dentro de blocos de código:

\begin{Shaded}
\begin{Highlighting}[]
\InformationTok{\textasciigrave{}\textasciigrave{}\textasciigrave{}\{r\}}
\InformationTok{knitr::kable(cars[1:5,], caption = "Carros")}
\InformationTok{\textasciigrave{}\textasciigrave{}\textasciigrave{}}
\end{Highlighting}
\end{Shaded}

\begin{Shaded}
\begin{Highlighting}[]
\NormalTok{knitr}\OperatorTok{::}\KeywordTok{kable}\NormalTok{(cars[}\DecValTok{1}\OperatorTok{:}\DecValTok{5}\NormalTok{,], }\DataTypeTok{caption =} \StringTok{"Carros"}\NormalTok{)}
\end{Highlighting}
\end{Shaded}

\begin{table}

\caption{\label{tab:unnamed-chunk-11}Carros}
\centering
\begin{tabular}[t]{r|r}
\hline
speed & dist\\
\hline
4 & 2\\
\hline
4 & 10\\
\hline
7 & 4\\
\hline
7 & 22\\
\hline
8 & 16\\
\hline
\end{tabular}
\end{table}

\begin{center}\rule{0.5\linewidth}{0.5pt}\end{center}

Exemplo utilizando outra \protect\hyperlink{linguagens-suportadas}{linguagem suportada}, neste caso python:

\begin{Shaded}
\begin{Highlighting}[]
\InformationTok{\textasciigrave{}\textasciigrave{}\textasciigrave{}\{python\}}
\InformationTok{print(5 ** 3)}
\InformationTok{\textasciigrave{}\textasciigrave{}\textasciigrave{}}
\end{Highlighting}
\end{Shaded}

\begin{Shaded}
\begin{Highlighting}[]
\BuiltInTok{print}\NormalTok{(}\DecValTok{5} \OperatorTok{**} \DecValTok{3}\NormalTok{)}
\end{Highlighting}
\end{Shaded}

\begin{verbatim}
## 125
\end{verbatim}

\begin{center}\rule{0.5\linewidth}{0.5pt}\end{center}

Exemplos mais detalhados de R e de outras linguagens suportadas estão disponíveis neste \href{exemplos/104-intro_blocos_de_codigo.html}{link}.

\hypertarget{customizauxe7uxe3o-e-tipos-de-arquivo}{%
\chapter{Customização e Tipos de Arquivo}\label{customizauxe7uxe3o-e-tipos-de-arquivo}}

\hypertarget{opuxe7uxf5es-avanuxe7adas-no-preuxe2mbulo}{%
\section{Opções avançadas no preâmbulo}\label{opuxe7uxf5es-avanuxe7adas-no-preuxe2mbulo}}

\hypertarget{apresentauxe7uxe3o-de-slides}{%
\section{Apresentação de slides}\label{apresentauxe7uxe3o-de-slides}}

\hypertarget{ioslides}{%
\subsection{ioslides}\label{ioslides}}

Para criar uma apresentação ioslides de R Markdown, você especifica \texttt{ioslides\_presentation} no output nos metadados YAML do seu documento. Você pode criar uma apresentação de slides dividida em seções usando as tags de cabeçalho \texttt{\#} e \texttt{\#\#} (você também pode criar um novo slide sem cabeçalho usando uma régua horizontal (\texttt{-\/-\/-}). Por exemplo, aqui está uma apresentação de slides simples:

\begin{verbatim}
---
title: "Curso de R Makrdown"
author: Departamento de Estatística
date: 12 de Agosto de 2020
output: ioslides_presentation
---

# Capítulo 1

## O que é?

- R
- RMarkdown

## Capítulo 2

- R
  - RMarkdown
\end{verbatim}

Você também pode adicionar um subtítulo ou seção incluíndo o texto após o caractere da barra vertifical (\texttt{\textbar{}}). Por exemplo:

\begin{verbatim}
## Isto é um texto | Exemplo de subtítulo ou seção
\end{verbatim}

\hypertarget{modos-de-exibiuxe7uxe3o-do-slide}{%
\subsection{Modos de Exibição do Slide}\label{modos-de-exibiuxe7uxe3o-do-slide}}

Veja a seguir um conjunto de atalhos no teclado que permitem modos de exibição alternativos:
- \texttt{f}: habilita o modo de tela cheia
- \texttt{w}: habilita o modo janela
- \texttt{o}: habilita o modo de visão geral
- \texttt{h}: habilita o modo de highlight do código
- \texttt{p}: exibe as anotações presentes

Pressione a tecla \texttt{Esc} para sair dos modos de exibição escolhidos.

\hypertarget{aparuxeancia-visual}{%
\subsection{Aparência Visual}\label{aparuxeancia-visual}}

\hypertarget{tamanho-da-apresentauxe7uxe3o}{%
\subsubsection*{Tamanho da Apresentação}\label{tamanho-da-apresentauxe7uxe3o}}
\addcontentsline{toc}{subsubsection}{Tamanho da Apresentação}

Você pode exibir a apresentação usando um wider form ao utilizar a opção \texttt{widescreen}. Você pode especificar que um texto menor seja usado com a opção \texttt{smaller}. Por exemplo:

\begin{verbatim}
---
output:
  ioslides_presentation:
    widescreen: true
    smaller: true
---
\end{verbatim}

Você também pode habilitar a opção \texttt{smaller} como uma opção slide-by-slide adicionando o atributo \texttt{.smaller} no cabeçario do slide:

\begin{verbatim}
## Este é um exemplo {.smaller}
\end{verbatim}

\hypertarget{velocidade-de-transiuxe7uxe3o}{%
\subsubsection*{Velocidade de Transição}\label{velocidade-de-transiuxe7uxe3o}}
\addcontentsline{toc}{subsubsection}{Velocidade de Transição}

Você pode customizar a velocidade de transição de um slide usando a opção \texttt{transition}. Isto pode ser \texttt{"default"}, \texttt{"slower"}, \texttt{"faster"}, ou um valor número com os números em segundos (por exemplo, \texttt{0.10}). Um exemplo:

\begin{verbatim}
---
output:
  ioslides_presentatiom:
    transition: slower
---
\end{verbatim}

\hypertarget{destacando-o-cuxf3digo}{%
\paragraph{Destacando o código}\label{destacando-o-cuxf3digo}}

É possível selecionar subconjuntos de código para ênfase adicional, adicionando um comentário especial de ``destaque'' ao redor do código. Por exemplo:

\begin{verbatim}
### <b>
x <- 10
y <- x * 2
### </b>
\end{verbatim}

A região destacada será exibida com uma fonte em negrito. Quando você quiser ajudar o público a se concentrar exclusivamente na região destacada, pressione a tecla \texttt{h} o resto do código desaparecerá.

\hypertarget{adicionando-um-logotipo}{%
\paragraph{Adicionando um Logotipo}\label{adicionando-um-logotipo}}

Você pode adicionar um logotipo à apresentação usando a opção \texttt{logo}(por padrão, o logotipo será exibido em um quadrado de 85x85 pixels). Por exemplo:

\begin{verbatim}
---
output:
  ioslides_presentation:
    logo: logo.png
---
\end{verbatim}

O gráfico do logotipo será redimensionado para 85x85 (se necessário) e adicionado ao slide de título. Uma versão de ícone menor do logotipo será incluída no rodapé esquerdo de cada slide.
O logotipo na página de título e o elemento retangular que o contém podem ser personalizados com CSS. Por exemplo:

\begin{verbatim}
.gdbar img {
  width: 300px !important;
  height: 150px !important;
  margin: 8px 8px;
}

.gdbar {
  width: 400px !important;
  height: 170px !important;
}
\end{verbatim}

Esses seletores devem ser colocados no arquivo de texto CSS.
Da mesma forma, o logotipo no rodapé de cada slide pode ser redimensionado para qualquer tamanho desejado. Por exemplo:

\begin{verbatim}
slides > slide:not(.nobackground):before {
  width: 150px;
  height: 75px;
  background-size: 150px 75px;
}
\end{verbatim}

Isso fará com que o logotipo do rodapé tenha 150 por 75 pixels de tamanho.

\hypertarget{tabelas}{%
\paragraph{Tabelas}\label{tabelas}}

O modelo ioslides tem um estilo padrão atraente para tabelas, então você não deve hesitar em adicionar tabelas para apresentar conjuntos de informações mais complexos. Pandoc Markdown suporta várias sintaxes para definir tabelas, que são descritas no Manual do Pandoc.

\hypertarget{layout-avanuxe7ado}{%
\paragraph{Layout Avançado}\label{layout-avanuxe7ado}}

Você pode centralizar o conteúdo em um slide adicionando o \texttt{.flexbox} e \texttt{.vcenter} atributos para o título do slide.
Por exemplo:

\begin{verbatim}
## Código {.flexbox .vcenter}
\end{verbatim}

Você pode centralizar horizontalmente o conteúdo, encerrando-o em uma tag \texttt{div} com a classe \texttt{centered}. Por exemplo:

\begin{verbatim}
<div class="centered">
Este texto está centralizado.
</div>
\end{verbatim}

Você pode fazer um layout de duas colunas usando a classe \texttt{columns-2}. Por exemplo:

\begin{verbatim}
<div class="columns-2">
  ![](image.png)

  - Bullet 1
  - Bullet 2
  - Bullet 3
</div>
\end{verbatim}

Observe que o conteúdo fluirá pelas colunas, portanto, se você quiser ter uma imagem de um lado e o texto do outro, certifique-se de que a imagem tenha altura suficiente para forçar o texto para o outro lado do slide.

\hypertarget{cor-do-texto}{%
\paragraph{Cor do Texto}\label{cor-do-texto}}

Você pode colorir o conteúdo usando classes de cores básicas \texttt{red}, \texttt{blue}, \texttt{green}, \texttt{yellow}, e \texttt{gray} (ou variações deles, por exemplo, \texttt{red2}, \texttt{red3}, \texttt{blue2}, \texttt{blue3}, etc.). Por exemplo:

\begin{verbatim}
<div class="red2">
Este texto é vermelho
\end{verbatim}

\hypertarget{modo-de-apresentauxe7uxe3o}{%
\paragraph{Modo de Apresentação}\label{modo-de-apresentauxe7uxe3o}}

Uma janela separada do apresentador também pode ser aberta (ideal para quando você está apresentando em uma tela, mas tem outra tela que é particular para você). A janela permanece sincronizada com a janela principal da apresentação e também mostra as notas do apresentador e uma miniatura do próximo slide. Para ativar o modo de apresentador, adicione \texttt{?presentme=true} ao URL da apresentação. Por exemplo:

\begin{verbatim}
minha-apresentacao.html?presentme=true
\end{verbatim}

A janela do modo de apresentador será aberta e sempre reabrirá com a apresentação até que seja desativada com:

\begin{verbatim}
minha-apresentacao.html?presentme=false
\end{verbatim}

Para adicionar notas do apresentador a um slide, inclua-as dentro de uma seção ``notes'' \texttt{div}. Por exemplo:

\begin{verbatim}
<div class="notes">
Este é a minha *nota*.

- Pode conter markdown
</div>
\end{verbatim}

\hypertarget{impressuxe3o-e-output-em-pdf}{%
\paragraph{Impressão e Output em PDF}\label{impressuxe3o-e-output-em-pdf}}

Você pode imprimir uma apresentação ioslides a partir de navegadores que tenham um bom suporte para CSS de impressão (até o momento, o Google Chrome tem o melhor suporte). A impressão mantém a maioria dos estilos visuais da versão HTML da apresentação.
Para criar uma versão PDF de uma apresentação, você pode usar o menu \texttt{Print\ to\ PDF} do Google Chrome.

\hypertarget{templates-customizados}{%
\paragraph{Templates Customizados}\label{templates-customizados}}

Você pode substituir o modelo Pandoc subjacente usando a opção \texttt{template}:

\begin{verbatim}
---
title: "Curso de RMarkdown"
output:
  ioslides_presentation:
    template: quarterly-report.html
---
\end{verbatim}

No entanto, observe que o nível de personalização que pode ser alcançado é limitado em comparação com os modelos de outros formatos de saída, porque os slides são gerados por formatação personalizada escrita em Lua e, como tal, o modelo usado deve incluir a string \texttt{RENDERED\_SLIDES} como pode ser encontrado no arquivo de modelo padrão com o caminho \texttt{rmarkdown:::rmarkdown\_system\_file("rmd/ioslides/default.html")}

\hypertarget{slidy}{%
\subsection{Slidy}\label{slidy}}

Para criar uma apresentação Slidy de RMarkdown, você precisa especificar o \texttt{slidy\_presentation} no output nos metadados YAML do seu documento. Você pode criar uma apresentação de slides dividida em seções usando as tags de cabeçalho \texttt{\#\ \#} você também pode criar um novo slide sem cabeçalho usando uma régua horizontal (\texttt{-\/-\/-}). Por exemplo, aqui está uma apresentação de slides simples:

\begin{verbatim}
---
title: "Curso de R Makrdown"
author: Departamento de Estatística
date: 12 de Agosto de 2020
output: slidy_presentation
---

# Capítulo 1

## O que é?

- R
- RMarkdown

## Capítulo 2

- R
- RMakrdown
\end{verbatim}

\hypertarget{modos-de-exibiuxe7uxe3o}{%
\subsubsection*{Modos de Exibição}\label{modos-de-exibiuxe7uxe3o}}
\addcontentsline{toc}{subsubsection}{Modos de Exibição}

Veja a seguir um conjunto de atalhos no teclado que permitem modos de exibição alternativos:

\begin{itemize}
\tightlist
\item
  \texttt{\textquotesingle{}C\textquotesingle{}}: Exibe o indíce
\item
  \texttt{\textquotesingle{}F\textquotesingle{}}: Alterna a exibição do rodapé.
\item
  \texttt{\textquotesingle{}A\textquotesingle{}}: Alterna a exibição de slides atuais para todos os slides (útil para imprimir folhetos).
\item
  \texttt{\textquotesingle{}S\textquotesingle{}}: Dimunue o tamanho da Fonte.
\item
  \texttt{\textquotesingle{}B\textquotesingle{}}: Aumentya o tamanho da Fonte.
\end{itemize}

\hypertarget{tamanho-do-texto}{%
\subsubsection*{Tamanho do Texto}\label{tamanho-do-texto}}
\addcontentsline{toc}{subsubsection}{Tamanho do Texto}

Você pode usar a opção \texttt{font\_adjustment} para aumentar ou diminuir o tamanho da fonte padrão (por exemplo, \texttt{-1} ou \texttt{+1}) para toda a apresentação. Por exemplo:

\begin{verbatim}
---
output:
  slidy_presentation:
    font_adjustment: -1
---
\end{verbatim}

Se quiser diminuir o tamanho do texto em um slide especifíco, você pode usar o atributo de slide \texttt{.smaller}. Vejamos um exemplo:

\begin{verbatim}
## Isto é um exemplo{.smaller}
\end{verbatim}

Se quiser aumentar o tamanho do texto em um slide especifíco, você pode usar o atributo de slide \texttt{.bigger}. Vejamos um exemplo:

\begin{verbatim}
## Isto é um exemplo{.bigger}
\end{verbatim}

Também é possível ajustar manualmente o tamanho da fonte padrão durante a sua apresentação usando o \texttt{\textquotesingle{}S\textquotesingle{}} (smaller) e \texttt{B}(bigger).

\hypertarget{elementos-de-rodapuxe9}{%
\subsubsection*{Elementos de Rodapé}\label{elementos-de-rodapuxe9}}
\addcontentsline{toc}{subsubsection}{Elementos de Rodapé}

Você pode adicionar uma contagem regressiva ao rodapé de seus slides usando a opção \texttt{duration} (a duração é especificada em minutos). Por exemplo:

\begin{verbatim}
---
output:
  slidy_presentation:
    duration: 45
---
\end{verbatim}

Você também pode adicionar um texto personalizado ao rodapé (por exemplo, o nome da sua organização, universidade ou Copyright) usando a opção \texttt{footer}. Por exemplo:

\begin{verbatim}
---
output:
  slidy_presentation:
    footer: "Copyright (c) 2020, Curso de RMarkdown"
---
\end{verbatim}

\hypertarget{beamer}{%
\subsection{Beamer}\label{beamer}}

\hypertarget{temas}{%
\subsubsection*{Temas}\label{temas}}
\addcontentsline{toc}{subsubsection}{Temas}

Você pode especificar os temas do Beamer usando as opções \texttt{theme}, \texttt{colortheme}, e \texttt{fonttheme}. Por exemplo:

\begin{verbatim}
---
output:
  beamer_presentation:
    theme: "AnnArbor"
    colortheme: "dolphin"
    fonttheme: "structurebold"
---
\end{verbatim}

\hypertarget{nuxedvel-do-slide}{%
\subsubsection*{Nível do Slide}\label{nuxedvel-do-slide}}
\addcontentsline{toc}{subsubsection}{Nível do Slide}

A opção \texttt{slide\_level} define o nível de título que define slides individuais.. Por padrão, este é o nível de cabeçalho mais alto na hierarquia, seguido imediatamente pelo conteúdo, e não por outro cabeçalho, em algum lugar do documento. Este padrão pode ser sobrescrito especificando um explícito \texttt{slide\_level}:

\begin{verbatim}
---
output:
  beamer_presentation:
    slide_level: 2
---
\end{verbatim}

\hypertarget{powerpoint}{%
\subsection{PowerPoint}\label{powerpoint}}

Para criar uma apresentação do PowerPoint a partir do R Markdown, você especifica a opção \texttt{powerpoint\_presentation} no output nos metadados YAML do seu documento. Note por favor que este formato de output só está disponóivel na versão v1.9 do \textbf{rmakrdown} e requer a última versão do Pandoc 2.10.1. (O Pandoc é um conversor de documentos, como por exemplo de Markdown para HTML). Você pode verificar as versões dos seus pacotes de \textbf{rmakrdown} e Pandoc com o comando \texttt{packageVersion(\textquotesingle{}rmarkdown\textquotesingle{})} e \texttt{rmarkdown::pandoc\_version()} respectivamente, no R.
Abaixo está um exemplo rápido de um output PowerPoint:

\begin{verbatim}
---
title: "Curso de R Makrdown"
author: Departamento de Estatística
date: 12 de Agosto de 2020
output: powerpoint_presentation
---

# Capítulo 1

## O que é?

- R
- RMarkdown

## Capítulo 2

- R
- RMakrdown
\end{verbatim}

O nível de slide (isto é, o nível de título que define slides individuais) é determinado da mesma forma que uma apresentação Beamer, e você pode especificar um nível explícito por meio de \texttt{slide\_level} da opção sob \texttt{powerpoint\_presentation}. Você também pode iniciar um novo slide sem cabeçalho usando uma régua horizontal \texttt{-\/-\/-}

Você pode gerar a maioria dos elementos suportados pelo Markdown do Pandoc no output do PowerPoint, como texto em negrito / itálico, notas de rodapé, marcadores, expressões matemáticas LaTeX, imagens e tabelas, etc.

Observe que imagens e tabelas sempre serão colocadas em novos slides.Os únicos elementos que podem coexistir com uma imagem ou tabela em um slide são o cabeçalho do slide e a legenda da imagem / tabela. Quando você tem um parágrafo de texto e uma imagem no mesmo slide, a imagem será movida para um novo slide automaticamente. As imagens serão dimensionadas automaticamente para caber no slide e, se o tamanho automático não funcionar bem, você pode controlar manualmente os tamanhos das imagens: para imagens estáticas incluídas por meio da sintaxe Markdown \texttt{!{[}{]}()} você pode usar o \texttt{width} e/ou \texttt{height} atributos em um par de chaves após a imagem, por exemplo, \texttt{!{[}caption{]}(foo.png)\{width=40\%\}} para imagens geradas dinamicamente a partir de blocos de código R, você pode usar o bloco \texttt{fig.width} e \texttt{fig.height} para controlar os tamanhos.

\begin{verbatim}
:::::: {.columns}
::: {.column width="40%"}
Content of the left column.
:::

::: {.column width="60%"}
Content of the right column.
:::
::::::
\end{verbatim}

\hypertarget{templates-customizados-1}{%
\subsubsection*{Templates Customizados}\label{templates-customizados-1}}
\addcontentsline{toc}{subsubsection}{Templates Customizados}

Como documentos do Word você pode personalizar a aparência das apresentações do PowerPoint passando um documento de referência personalizado por meio da opção \texttt{reference\_doc}, por exemplo:

\begin{verbatim}
---
title: "Curso de RMarkdown"
output:
  powerpoint_presentation:
    reference_doc: curso-rmarkdown.pptx
---
\end{verbatim}

Observe que a opção \texttt{reference\_doc} requer a versão 1.9 ou superior do \textbf{rmarkdown}:

\begin{verbatim}
if (packageVersion('rmarkdown') <= '1.9') {
    install.packages('rmarkdown')  # atualiza o pacote rmarkdown do CRAN
}
\end{verbatim}

Basicamente, qualquer modelo incluído em uma versão recente do Microsoft PowerPoint deve funcionar. Você pode criar um novo arquivo \texttt{*.pptx} do menu do PowerPoint \texttt{Arquivos\ -\textgreater{}\ Novo} com o template desejado, salve o novo arquivo e use-o como documento de referência (template) através da opção \texttt{reference\_doc}. O Pandoc lerá os estilos no modelo e os aplicará à apresentação do PowerPoint a ser criada a partir do R Markdown.

\hypertarget{reveal.js}{%
\subsection{Reveal.js}\label{reveal.js}}

O pacote \textbf{revealjs} fornece um tipo de output no formato \texttt{revealjs::revealjs\_presentation} que pode ser usado para criar outro estilo de slides HTML5 com base na biblioteca JavaScript \textbf{reveal.js.} Você pode instalar o pacote R do CRAN:

\begin{verbatim}
install.packages("revealjs")
\end{verbatim}

Para criar uma apresentação reveal.js de R Markdown, você especifica \texttt{revealjs\_presentation} no output nos metadados YAML do seu documento. Você pode criar uma apresentação de slides dividida em seções usando as tags de cabeçalho \texttt{\#} e \texttt{\#\#} (você também pode criar um novo slide sem cabeçalho usando uma régua horizontal (\texttt{-\/-\/-}). Por exemplo, aqui está uma apresentação de slides simples:

\begin{verbatim}
---
title: "Curso de R Makrdown"
author: Departamento de Estatística
date: 12 de Agosto de 2020
output: revealjs::revealjs_presentation
---

# Capítulo 1

## O que é?

- R
- RMarkdown

## Capítulo 2

- R
- RMakrdown
\end{verbatim}

\hypertarget{xaringan}{%
\subsection{Xaringan}\label{xaringan}}

O pacote \textbf{xaringan} é uma extensão R Markdown baseada no JavaScript da biblioteca remark.js para gerar apresentações HTML5 de um estilo diferente.

O nome ``xaringan'' veio do Sharingan no mangá e anime japoneses ``Naruto''. A palavra foi deliberadamente escolhida para ser difícil de pronunciar para a maioria das pessoas (a menos que você tenha assistido ao anime), porque seu autor (eu) amava muito o estilo e estava preocupado que se tornasse muito popular. A preocupação era um tanto ingênua , porque o estilo é realmente muito personalizável e os usuários começaram a contribuir com mais temas para o pacote posteriormente.
O pacote \textbf{xaringan} é uma extensão R Markdown baseada no JavaScript da biblioteca remark.js; a biblioteca remark.js suporta apenas Markdown, e o xaringan adicionou o suporte para R Markdown, bem como outros utilitários para tornar mais fácil construir e visualizar slides. Possibilitando que você crie apresentações ninjas!.

\hypertarget{documentos}{%
\section{Documentos}\label{documentos}}

\hypertarget{documento-de-texto-opendocument}{%
\subsection{Documento de texto OpenDocument}\label{documento-de-texto-opendocument}}

Para criar um documento OpenDocument Text (ODT) a partir do R Markdown, você especifica o \texttt{odt\_document} no formato de saída nos metadados YAML do seu documento:

\begin{Shaded}
\begin{Highlighting}[]
\CommentTok{{-}{-}{-}}
\AnnotationTok{title:}\CommentTok{ "Curso"}
\AnnotationTok{author:}\CommentTok{ Estats}
\AnnotationTok{date:}\CommentTok{ Agosto 01, 2020}
\AnnotationTok{output:}\CommentTok{ odt\_document}
\CommentTok{{-}{-}{-}}
\end{Highlighting}
\end{Shaded}

Semelhante a \texttt{word\_document},você, também pode fornecer um documento de referência de estilo para \texttt{odt\_document} percorrer a \texttt{reference\_odt} na configuração. Para obter os melhores resultados, o documento ODT de referência deve ser uma versão modificada de um arquivo ODT produzido usando rmarkdown. Por exemplo:

\begin{Shaded}
\begin{Highlighting}[]
\CommentTok{{-}{-}{-}}
\AnnotationTok{title:}\CommentTok{ "Curso"}
\AnnotationTok{output:}
\CommentTok{  odt\_document:}
\CommentTok{    reference\_odt: meu{-}estilo.odt}
\CommentTok{{-}{-}{-}}
\end{Highlighting}
\end{Shaded}

\hypertarget{documento-do-word}{%
\subsection{Documento do Word}\label{documento-do-word}}

Para criar um documento do Word a partir do R Markdown, você especifica o \texttt{word\_document} no formato de saída nos metadados YAML do seu documento:

\begin{Shaded}
\begin{Highlighting}[]
\CommentTok{{-}{-}{-}}
\AnnotationTok{title:}\CommentTok{ "Curso"}
\AnnotationTok{author:}\CommentTok{ Estats}
\AnnotationTok{date:}\CommentTok{ Agosto 01, 2020}
\AnnotationTok{output:}\CommentTok{ word\_document}
\CommentTok{{-}{-}{-}}
\end{Highlighting}
\end{Shaded}

A característica mais notável dos documentos do Word é o modelo do Word, também conhecido como ``documento de referência de estilo''. Você pode especificar um documento a ser usado como referência de estilo na produção de um \texttt{*.docx} arquivo (um documento do Word). Isso permitirá que você personalize itens como margens e outras características de formatação. Para obter os melhores resultados, o documento de referência deve ser uma versão modificada de um \texttt{.docx} arquivo produzido usando rmarkdown. O caminho de tal documento pode ser passado para o \texttt{reference\_docx} argumento do \texttt{word\_document} no formato de saída. Passe \texttt{"default"} para usar os estilos padrão. Por exemplo:

\begin{Shaded}
\begin{Highlighting}[]
\CommentTok{{-}{-}{-}}
\AnnotationTok{title:}\CommentTok{ "Curso"}
\AnnotationTok{output:}
\CommentTok{  word\_document:}
\CommentTok{    reference\_docx: meu{-}estilo.docx}
\CommentTok{{-}{-}{-}}
\end{Highlighting}
\end{Shaded}

\hypertarget{documento-de-pdf}{%
\subsection{Documento de PDF}\label{documento-de-pdf}}

Para criar um documento PDF a partir do R Markdown, você especifica o \texttt{pdf\_document} no formato de saída nos metadados YAML:

\begin{Shaded}
\begin{Highlighting}[]
\CommentTok{{-}{-}{-}}
\AnnotationTok{title:}\CommentTok{ "Curso"}
\AnnotationTok{author:}\CommentTok{ Estats}
\AnnotationTok{date:}\CommentTok{ Agosto 01, 2020}
\AnnotationTok{output:}\CommentTok{ pdf\_document}
\CommentTok{{-}{-}{-}}
\end{Highlighting}
\end{Shaded}

Dentro dos documentos R Markdown que geram saída em PDF, você pode usar LaTeX bruto e até mesmo definir macros LaTeX.

Observe que a saída de PDF (incluindo slides Beamer) requer uma instalação do LaTeX:

\begin{Shaded}
\begin{Highlighting}[]
\NormalTok{install.packages(\textquotesingle{}tinytex\textquotesingle{})}
\NormalTok{tinytex::install\_tinytex()  \# install TinyTeX}
\end{Highlighting}
\end{Shaded}

\hypertarget{uxedndice}{%
\subsubsection{Índice}\label{uxedndice}}

Você pode adicionar um sumário usando a \texttt{toc} na cofiguração e especificar a profundidade dos cabeçalhos aos quais se aplica usando a \texttt{toc\_depth} na cofiguração. Por exemplo:

\begin{Shaded}
\begin{Highlighting}[]
\CommentTok{{-}{-}{-}}
\AnnotationTok{title:}\CommentTok{ "Curso"}
\AnnotationTok{output:}
\CommentTok{  pdf\_document:}
\CommentTok{    toc: true}
\CommentTok{    toc\_depth: 2}
\CommentTok{{-}{-}{-}}
\end{Highlighting}
\end{Shaded}

Se a profundidade do TOC não for especificada explicitamente, o padrão é 2 (significando que todos os cabeçalhos de nível 1 e 2 serão incluídos no TOC), enquanto o padrão é 3 pol \texttt{html\_document}.

Você pode adicionar numeração de seção aos cabeçalhos usando a \texttt{number\_sections} na cofiguração:

\begin{Shaded}
\begin{Highlighting}[]
\CommentTok{{-}{-}{-}}
\AnnotationTok{title:}\CommentTok{ "Curso"}
\AnnotationTok{output:}
\CommentTok{  pdf\_document:}
\CommentTok{    toc: true}
\CommentTok{    number\_sections: true}
\CommentTok{{-}{-}{-}}
\end{Highlighting}
\end{Shaded}

Se você estiver familiarizado com LaTeX, \texttt{number\_sections:\ true} significa \texttt{\textbackslash{}section\{\}} e \texttt{number\_sections:\ false}significa \texttt{\textbackslash{}section*\{\}} para seções em LaTeX (também se aplica a outros níveis de ``seções'' como \texttt{\textbackslash{}chapter\{\}}, e \texttt{\textbackslash{}subsection\{\}}).

\hypertarget{opuxe7uxf5es-de-imagem}{%
\subsubsection{Opções de imagem}\label{opuxe7uxf5es-de-imagem}}

Existem várias opções que afetam a saída de figuras em documentos PDF:

\begin{itemize}
\item
  \texttt{fig\_width} e \texttt{fig\_height} pode ser usado para controlar a largura e altura da figura padrão.
\item
  \texttt{fig\_crop} controla se o \texttt{pdfcrop} utilitário, se disponível em seu sistema, é aplicado automaticamente a figuras em PDF.
\item
  \texttt{fig\_caption} controla se as figuras são renderizadas com legendas.
\item
  \texttt{dev} controla o dispositivo gráfico usado para renderizar figuras.
\end{itemize}

\begin{Shaded}
\begin{Highlighting}[]
\CommentTok{{-}{-}{-}}
\AnnotationTok{title:}\CommentTok{ "Curso"}
\AnnotationTok{output:}
\CommentTok{  pdf\_document:}
\CommentTok{    fig\_width: 7}
\CommentTok{    fig\_height: 6}
\CommentTok{    fig\_caption: true}
\CommentTok{{-}{-}{-}}
\end{Highlighting}
\end{Shaded}

\hypertarget{impressuxe3o-de-tabela}{%
\subsubsection{Impressão de Tabela}\label{impressuxe3o-de-tabela}}

Você pode aprimorar a exibição padrão da tabela por meio da \texttt{df\_print} na configuração. Os valores válidos são apresentados na Tabela abaixo:

\begin{longtable}[]{@{}ll@{}}
\toprule
Opção & Descrição\tabularnewline
\midrule
\endhead
default & Chame o \texttt{print.data.frame} método genérico\tabularnewline
kable & Use a \texttt{knitr::kable()} função\tabularnewline
tibble & Use a \texttt{tibble::print.tbl\_df()} função\tabularnewline
\bottomrule
\end{longtable}

Por Exemplo:

\begin{Shaded}
\begin{Highlighting}[]
\CommentTok{{-}{-}{-}}
\AnnotationTok{title:}\CommentTok{ "Curso"}
\AnnotationTok{output:}
\CommentTok{  pdf\_document:}
\CommentTok{    df\_print: kable}
\CommentTok{{-}{-}{-}}
\end{Highlighting}
\end{Shaded}

\hypertarget{destaque-de-sintaxe}{%
\subsubsection{Destaque de sintaxe}\label{destaque-de-sintaxe}}

A \texttt{highlight} na configuração especifica o estilo de realce da sintaxe. Seu uso em \texttt{pdf\_document} é o mesmo que \texttt{html\_document}. Por exemplo:

\begin{Shaded}
\begin{Highlighting}[]
\CommentTok{{-}{-}{-}}
\AnnotationTok{title:}\CommentTok{ "Curso"}
\AnnotationTok{output:}
\CommentTok{  pdf\_document:}
\CommentTok{    highlight: tango}
\CommentTok{{-}{-}{-}}
\end{Highlighting}
\end{Shaded}

\hypertarget{opuxe7uxf5es-latex}{%
\subsubsection{Opções LaTeX}\label{opuxe7uxf5es-latex}}

Muitos aspectos do modelo de látex usado para criar documentos PDF podem ser personalizados usando top-level YAML metadados (note que estas opções não aparecem por baixo da \texttt{output} secção, mas sim aparecer no nível superior, juntamente com \texttt{title}, \texttt{author} e assim por diante). Por exemplo:

\begin{Shaded}
\begin{Highlighting}[]
\CommentTok{{-}{-}{-}}
\AnnotationTok{title:}\CommentTok{ "Curso"}
\AnnotationTok{output:}\CommentTok{ pdf\_document}
\AnnotationTok{fontsize:}\CommentTok{ 11pt}
\AnnotationTok{geometry:}\CommentTok{ margin=1in}
\CommentTok{{-}{-}{-}}
\end{Highlighting}
\end{Shaded}

Algumas variáveis de metadados disponíveis são exibidas na Tabela abaixo:

\begin{longtable}[]{@{}ll@{}}
\toprule
\begin{minipage}[b]{0.47\columnwidth}\raggedright
Variável\strut
\end{minipage} & \begin{minipage}[b]{0.47\columnwidth}\raggedright
Descrição\strut
\end{minipage}\tabularnewline
\midrule
\endhead
\begin{minipage}[t]{0.47\columnwidth}\raggedright
lang\strut
\end{minipage} & \begin{minipage}[t]{0.47\columnwidth}\raggedright
Código de idioma do documento \textbar{}\strut
\end{minipage}\tabularnewline
\begin{minipage}[t]{0.47\columnwidth}\raggedright
fontsize\strut
\end{minipage} & \begin{minipage}[t]{0.47\columnwidth}\raggedright
Tamanho da fonte (por exemplo, \texttt{10pt}, \texttt{11pt}, ou \texttt{12pt})\strut
\end{minipage}\tabularnewline
\begin{minipage}[t]{0.47\columnwidth}\raggedright
documentclass\strut
\end{minipage} & \begin{minipage}[t]{0.47\columnwidth}\raggedright
Classe de documento LaTeX (por exemplo, \texttt{article})\strut
\end{minipage}\tabularnewline
\begin{minipage}[t]{0.47\columnwidth}\raggedright
classoption\strut
\end{minipage} & \begin{minipage}[t]{0.47\columnwidth}\raggedright
Opções para documentclass (por exemplo, \texttt{oneside})\strut
\end{minipage}\tabularnewline
\begin{minipage}[t]{0.47\columnwidth}\raggedright
geometry\strut
\end{minipage} & \begin{minipage}[t]{0.47\columnwidth}\raggedright
Opções para geometria e class(por exemplo, \texttt{margin=1in})\strut
\end{minipage}\tabularnewline
\begin{minipage}[t]{0.47\columnwidth}\raggedright
mainfont, sansfont, monofont, mathfont\strut
\end{minipage} & \begin{minipage}[t]{0.47\columnwidth}\raggedright
Fontes de documentos (funciona apenas com \texttt{xelatexe} \texttt{lualatex})\strut
\end{minipage}\tabularnewline
\begin{minipage}[t]{0.47\columnwidth}\raggedright
linkcolor, urlcolor, citecolor\strut
\end{minipage} & \begin{minipage}[t]{0.47\columnwidth}\raggedright
Cor para links internos, externos e de citação\strut
\end{minipage}\tabularnewline
\bottomrule
\end{longtable}

\hypertarget{pacotes-latex-para-citauxe7uxf5es}{%
\subsubsection{Pacotes LaTeX para citações}\label{pacotes-latex-para-citauxe7uxf5es}}

Por padrão, as citações são processadas por meio do \texttt{pandoc-citeproc}, o que funciona para todos os formatos de saída. Para saída em PDF, às vezes é melhor usar pacotes LaTeX para processar citações, como \texttt{natbib} ou \texttt{biblatex}. Para usar um desses pacotes, basta definir a opção \texttt{citation\_package} como \texttt{natbib} ou \texttt{biblatex}, por exemplo:

\begin{Shaded}
\begin{Highlighting}[]
\CommentTok{{-}{-}{-}}
\AnnotationTok{output:}
\CommentTok{  pdf\_document:}
\CommentTok{    citation\_package: natbib}
\CommentTok{{-}{-}{-}}
\end{Highlighting}
\end{Shaded}

\hypertarget{html}{%
\section{HTML}\label{html}}

\hypertarget{opuxe7uxf5es-de-preuxe2mbulo}{%
\subsection{Opções de preâmbulo:}\label{opuxe7uxf5es-de-preuxe2mbulo}}

\begin{itemize}
\tightlist
\item
  highlight
\item
  toc
\item
  dev
\item
  extra\_dependencies
\item
  toc\_depth
\item
  df\_print
\item
  css
\item
  toc\_float
\item
  code\_folding
\item
  includes
\item
  number\_sections
\item
  code\_download
\item
  keep\_md
\item
  section\_divs
\item
  self\_contained
\item
  lib\_dir
\item
  fig\_width
\item
  theme
\item
  md\_extensions=
\item
  fig\_height
\item
  highlight
\item
  pandoc\_args
\item
  fig\_retina
\item
  mathjax
\item
  fig\_caption
\item
  template
\end{itemize}

\hypertarget{pacotes-uxfateis}{%
\section{Pacotes úteis}\label{pacotes-uxfateis}}

Um dos pacotes úteis do RMarkdown é o flexdashboars.
Esse pacote possui várias funções como:
- Publicar um grupo de visualizações de dados como um painel;
- Incorporar uma ampla variedade de componentes, incluindo widgets HTML, gráficos R, dados tabulares, medidores, caixas de valor e anotações de texto.
-Especificar layouts baseados em linha ou coluna (os componentes são redimensionados de forma inteligente para preencher o navegador e adaptados para exibição em dispositivos móveis).
- Crie storyboards para apresentar sequências de visualizações e comentários relacionados;
-Opcionalmente, use o Shiny para gerar visualizações dinamicamente.

Para criar um painel no RMarkdown, existem duas opções:
A primeira é criar um documento em RMarkdown com flexdashboard::flex\_dashboard como formato de saída.
\includegraphics{img/Dash 1.png}
Para a segunda opção é necessário primeiro instalar o pacote ``flexdashboard'' no RStudio, depois siga os passos abaixo:
File -\textgreater{} New File -\textgreater{} RMarkdown
Depois:
\includegraphics{img/Dash 2.png}

\hypertarget{referuxeancia}{%
\chapter{Referência}\label{referuxeancia}}

\textless!DOCTYPE html\textgreater{}

105-referencia.utf8

\hypertarget{header}{}
\begin{fluid-row}

\end{fluid-row}

\begin{columns}

\begin{column}

Sintaxe

\end{column}

\begin{column}

~

\end{column}

\begin{column}

Resultado

\end{column}

\end{columns}

\begin{columns}

\begin{column}

Texto

\end{column}

\begin{column}

~

\end{column}

\begin{column}

Texto

\end{column}

\end{columns}

\begin{columns}

\begin{column}

Termine uma linha com dois espaços para
começar um novo parágrafo.

\end{column}

\begin{column}

~

\end{column}

\begin{column}

Termine uma linha com dois espaços para começar um novo parágrafo.

\end{column}

\end{columns}

\begin{columns}

\begin{column}

\emph{Itálico} e \emph{itálico}

\end{column}

\begin{column}

~

\end{column}

\begin{column}

Itálico e itálico

\end{column}

\end{columns}

\begin{columns}

\begin{column}

\textbf{Negrito} e \textbf{negrito}

\end{column}

\begin{column}

~

\end{column}

\begin{column}

Negrito e negrito

\end{column}

\end{columns}

\begin{columns}

\begin{column}

Sobrescrito\textsuperscript{2}

\end{column}

\begin{column}

~

\end{column}

\begin{column}

Sobrescrito2

\end{column}

\end{columns}

\begin{columns}

\begin{column}

\sout{Texto taxado}

\end{column}

\begin{column}

~

\end{column}

\begin{column}

Texto taxado

\end{column}

\end{columns}

\begin{columns}

\begin{column}

\href{des.uem.br}{Link}

\end{column}

\begin{column}

~

\end{column}

\begin{column}

Link

\end{column}

\end{columns}

\begin{columns}

\begin{column}

\# Cabeçalho 1

\end{column}

\begin{column}

~

\end{column}

\begin{column}

\hypertarget{cabeuxe7alho-1}{}
\begin{section}

Cabeçalho 1

\end{section}

\end{column}

\end{columns}

\begin{columns}

\begin{column}

\#\# Cabeçalho 2

\end{column}

\begin{column}

~

\end{column}

\begin{column}

\hypertarget{cabeuxe7alho-2}{}
\begin{section}

Cabeçalho 2

\end{section}

\end{column}

\end{columns}

\begin{columns}

\begin{column}

\#\#\# Cabeçalho 3

\end{column}

\begin{column}

~

\end{column}

\begin{column}

\hypertarget{cabeuxe7alho-3}{}
\begin{section}

Cabeçalho 3

\end{section}

\end{column}

\end{columns}

\begin{columns}

\begin{column}

\#\#\#\# Cabeçalho 4

\end{column}

\begin{column}

~

\end{column}

\begin{column}

\hypertarget{cabeuxe7alho-4}{}
\begin{section}

Cabeçalho 4

\end{section}

\end{column}

\end{columns}

\begin{columns}

\begin{column}

\#\#\#\#\# Cabeçalho 5

\end{column}

\begin{column}

~

\end{column}

\begin{column}

\hypertarget{cabeuxe7alho-5}{}
\begin{section}

Cabeçalho 5

\end{section}

\end{column}

\end{columns}

\begin{columns}

\begin{column}

\#\#\#\#\#\# Cabeçalho 6

\end{column}

\begin{column}

~

\end{column}

\begin{column}

\hypertarget{cabeuxe7alho-6}{}
\begin{section}

Cabeçalho 6

\end{section}

\end{column}

\end{columns}

\begin{columns}

\begin{column}

Expressão matemática:
\(f_x(x) = \lambda e^{- \lambda x}\)

\end{column}

\begin{column}

~

\end{column}

\begin{column}

Expressão matemática: {\(f_x(x) = \lambda e^{- \lambda x}\)}

\end{column}

\end{columns}

\begin{columns}

\begin{column}

Expressão matemática em bloco:

\[X = \begin{bmatrix}1 &amp; x_{1}\\</code><br />
<code>1 &amp; x_{2}\\</code><br />
<code>1 &amp; x_{3}</code><br />
<code>\end{bmatrix}\]

\end{column}

\begin{column}

~

\end{column}

\begin{column}

Expressão matemática em bloco:

{\[X = \begin{bmatrix}1 &amp; x_{1}\\
1 &amp; x_{2}\\
1 &amp; x_{3}
\end{bmatrix}\]}

\end{column}

\end{columns}

\begin{columns}

\begin{column}

Imagem: \includegraphics{http://tny.im/lTZ}

\end{column}

\begin{column}

~

\end{column}

\begin{column}

Imagem:

\end{column}

\end{columns}

\begin{column}

Linha horizontal:

***

\end{column}

\begin{column}

~

\end{column}

Linha horizontal:

\begin{columns}

\begin{column}

Exemplo de nota de rodapé\footnote{Nota de rodapé}

\end{column}

\begin{column}

~

\end{column}

\begin{column}

Exemplo de nota de rodapé1

\end{column}

\end{columns}

\begin{columns}

\begin{column}

\end{column}

\begin{column}

~

\end{column}

\begin{column}

item

item

item

item

item

item

item

item

item

\end{column}

\end{columns}

\begin{columns}

\begin{column}

\end{column}

\begin{column}

~

\end{column}

\begin{column}

item 1

item 2

item 3

item 1

item 2

\end{column}

\end{columns}

\begin{columns}

\begin{column}

\end{column}

\begin{column}

~

\end{column}

\begin{column}

``É notável uma ciência que começou com jogos de azar tenha se tornado o mais importante objeto do conhecimento humano.''

--- Pierre Simon Laplace

\end{column}

\end{columns}

\begin{columns}

\begin{column}

\end{column}

\begin{column}

~

\end{column}

\begin{column}

\end{column}

\end{columns}

\begin{columns}

\begin{column}

\end{column}

\begin{column}

~

\end{column}

\begin{column}

Coluna 1

Coluna 2

11

12

21

22

\end{column}

\end{columns}

\begin{columns}

\begin{column}

Rodando código \texttt{r\ “no\ meio\ do\ texto”}

\end{column}

\begin{column}

~

\end{column}

\begin{column}

Rodando código no meio do texto

\end{column}

\end{columns}

\begin{columns}

\begin{column}

\end{column}

\begin{column}

~

\end{column}

\begin{column}

\end{column}

\end{columns}

\begin{columns}

\begin{column}

\end{column}

\begin{column}

~

\end{column}

\begin{column}

\end{column}

\end{columns}

\begin{columns}

\begin{column}

\end{column}

\begin{column}

~

\end{column}

\begin{column}

\end{column}

\end{columns}

\begin{columns}

\begin{column}

\end{column}

\begin{column}

~

\end{column}

\begin{column}

Cars

speed

dist

4

2

4

10

7

4

7

22

8

16

\end{column}

\end{columns}

\begin{columns}

\begin{column}

\end{column}

\begin{column}

~

\end{column}

\begin{column}

\end{column}

\end{columns}

Nota de rodapé↩︎

--\textgreater{}

\end{document}
